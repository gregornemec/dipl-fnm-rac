% LaTeX Template
% This template is made for project reports
% 	You may adjust it to your own needs/purposesF

% Copyright: http://www.howtotex.com/
% Date: March 2011

%%% Preamble
\documentclass[12pt, a4paper, oneside, slovene]{article}
%\documentclass[paper=a4, fontsize=12pt]{scrartcl}	% Article class of KOMA-script with 11pt font and a4 format

\usepackage[a4paper, left=2.5cm, right=2.5cm, top=2.5cm, bottom=2.5cm]{geometry}
\usepackage[T1]{fontenc}
\usepackage{fourier}

\usepackage[utf8x]{inputenc}
\usepackage[slovene]{babel}															% English language/hyphenation
\usepackage[protrusion=true,expansion=true]{microtype}				% Better typography
\usepackage{amsmath,amsfonts,amsthm}										% Math packages
%\usepackage[pdftex]{graphicx}
\usepackage{graphicx}
% Enable pdflatex

\usepackage{url}
\usepackage{hyperref}

\usepackage{lscape}

%%% Custom sectioning (sectsty package)
\usepackage{sectsty}
% Custom sectioning (see below)
\allsectionsfont{\normalfont\scshape}	% Change font of al section commands

\usepackage{setspace}

%%% Custom headers/footers (fancyhdr package)
\usepackage{fancyhdr}
\pagestyle{fancyplain}
\fancyhead{}														% No page header
% \fancyfoot[L]{\small \url{HowToTeX.com}}		% You may
\fancyfoot[L]{}		% You may remove/edit this line
\fancyfoot[C]{}			% Empty
\fancyfoot[R]{\thepage}									% Pagenumbering
\renewcommand{\headrulewidth}{0pt}			% Remove header underlines
\renewcommand{\footrulewidth}{0pt}				% Remove footer underlines
%\setlength{\headheight}{13.6pt}


%%% Equation and float numbering
%\numberwithin{equation}{section}		% Equationnumbering: section.eq#
%\numberwithin{figure}{section}			% Figurenumbering: section.fig#
%\numberwithin{table}{section}		        % Tablenumbering: section.tab#

%Indent set to 0pt
\setlength\parindent{0pt}

%Nastavitve odstavkov
\setlength{\parindent}{0cm}
\setlength{\parskip}{10px}

%Prostor med vrsticami
%\singlespacing
\onehalfspacing


%%% Maketitle metadata
\newcommand{\horrule}[1]{\rule{\linewidth}{#1}} 	% Horizontal rule
%%% Begin document


%Spremenljivke

%Ustanova
\newcommand{\univerza}{UNIVERZA V MARIBORU}
\newcommand{\fakulteta}{FAKULTETA ZA NARAVOSLOVJE IN MATEMATIKO}
\newcommand{\naslovfak}{Koroška cesta 160, 2000 Maribor}
\newcommand{\oddelek}{Oddelek za matematiko in računalništvo}

%Naslov
\newcommand{\naslov}{DIPLOMSKO DELO} \newcommand{\podnaslov}{Spletni
  portali za učenje programiranja: klasifikacija in možnosti uporabe v
  izobraževanj}

% Mentor/profesor
\newcommand{\mentor}{doc. dr. Igor Pesek}

% Študent
\newcommand{\kandidat}{Gregor Nemec}
\newcommand{\program}{Fizika in računalništvo}
\newcommand{\letnik}{4.letnik}
\newcommand{\smer}{Fizika in računalništvo}
\newcommand{\email}{gregorneme@gmail.com}

% Naslov, datum

\newcommand{\datum}{Maribor, 2015}

%Začetek dokumenta

\begin{document}

\thispagestyle{empty}
%%% Document title variables

%%%%%%%%%%%%%%%%%%%%%%%%%%%%%%%%%%%%%%%%%%%%%%%%%%%%%%%%%%%%%% 
% NASLOVNICA
%%%%%%%%%%%%%%%%%%%%%%%%%%%%%%%%%%%%%%%%%%%%%%%%%%%%%%%%%%%%%% 
\pagestyle{empty}
% ---------======== Naslovnica =======----------
\begin{center}
  \large
  \UNIVERZA

  \FAKULTETA

  \oddelek

\end{center}

\vspace{4cm}


\begin{center}
  \vspace{1cm}
  {\Huge \bfseries \naslov}\\
  \vspace{2cm}
  {\Large \kandidat}
\end{center}
\vspace*{\fill}
\vspace*{\fill}

% \begin{flushleft}
%   % \vspace{1cm}
%   \Large \mentor
% \end{flushleft}

\begin{center}


  \vspace{2cm}


  {\large \datum}
\end{center}


% ---------======== Naslovnica =======----------

%%% Local Variables:
%%% mode: latex
%%% TeX-master: "diploma"
%%% End:

%\newpage
%\tableofcontents
\setcounter{page}{1}
\newpage
%% UVOD je posebej datoteka zaradi lažjega upravljanja.

\section{Uporaba računalnika v izobraževanju}
\label{sec:uporaba-raunalnika-v}

Po modelu uporabe računalnika v izobraževanju, njegova uporaba pri
učenju programiranja spada v primarno področje, saj sem prištevamo
aktivnosti s katerimi želimo uporabnike seznaniti z delovanjem in
uporabo računalnika oz. sodobno informacijsko komunikacijsko
tehnologijo (\textbf{IKT}) \cite{model_uporabe_rac_izo-web}. Računalnik
je tista učna vsebina, ki jo obravnavamo. Računalništvo v osnovi
nastopa v dveh pomembnih področjih \cite{gerlic_2000}:

\begin{itemize}
\item kot element splošne izobrazbe,
\item kot element ožje strokovne - poklicne izobrazbe
  oz. usposabljanja.
\end{itemize}

V današnjem času se računalnik kot element splošne izobrazbe kaže kot
velika potreba oz. se zdi znanje njegove uporabe samoumevno. Že pri
najmlajših otrocih računalnik vzbuja zanimanje in interes.  Računalnik
je postal intelektualno orodje in pripomoček v vsaki sferi človekove
dejavnosti in je prodrl tudi v šolo. Tako imenovana \emph{računalniška
  pismenost} postaja nuja in zajema vse to kar bi človek moral znati o
računalniku in to, kako je potrebno z njim delati, da bo uspešno živel
v družbi, ki je osnovana na informacijah (informacijski družbi)
\cite{klemencic_2011}.

%% Kje je programiranje, ali spada v strokovno izobraževanje in ne
%% spada v splošno računalniško pismenost, ali je danes zahteva po
%% zanju programiranju le to že uvrstila med splošno pismenost.


%% Podrobneje kje se izvaja računalniško izobraževanje v OŠ in SŠ
%% imamo v posebej zato namenjenem poglavju Računalništvo se ne izvaja
%% kot redni predmet, temveč poteka v


\subsection{Zgodovin uporabe računalnika v izobraževanju
računalništva.}\label{zgodovina_uporabe_racunalnika_v_izobrazevanju}

%Splošna zgodovina uporabe računalništva v izobraževanju
%Razdeljeno imam zgodovino rac. v izobrazevanju in programiranje v
%izo, ali je naj vse skupaj.

\subsection{Pregled aplikacij v izobraževanju računalništva.}
\label{pregled-aplikavij-v-izobraux17eevanju-raux10dunalniux161tva.}



\subsubsection{\texorpdfstring{Sisatemi CAI ( \emph{ang Computer
      Assisted Instruction} )}{Sisatemi CAI ( ang Computer Assisted
    Instruction )}}
\label{sisatemi-cai-ang-computer-assisted-instruction}

Računalniško podprt sistem za učenje.

%-\textgreater{} Točna definicija gerlič.

Katere stvari mora nujno posedovati orodje za učenje novincem, da jim je
kar se da v največjo pomoč?

Pomembna sta dva glavna cilja na katerih temelji učenje programerjev
novincev:

\begin{enumerate}
\def\labelenumi{\arabic{enumi}.}
%\tightlist
\item
  Sistemi za učenje, pomagajo izključno učenju programiranja.
\item
  Sistemi za pomoč pri programiranju, ko želimo programiranje uporabiti
  za dosego nekega drugega cilja.
\end{enumerate}

\begin{itemize}
%\tightlist
\item
  Novice programming system and languge taxonomy - tabela (str. 43).
\end{itemize}

Skupine programskih orodij, ki so v pomoč novincem so naslednje:

\begin{enumerate}
\def\labelenumi{\arabic{enumi}.}
%\tightlist
\item
  mikro svetovi,
\item
  vizualna okolja za programiranje,
\item
  okolje za izdelavo modela poteka,
\item
  okolje za izdelovanje objektov,
\item
  okolje za risanje in realizacijo algoritmov.
\end{enumerate}

\subsubsection{\texorpdfstring{Sistemi CAA ( \emph{ang. Computer
      Assisted assessment} ).}{Sistemi CAA ( ang. Computer Assisted
    assessment ).}}
\label{sistemi-caa-ang.-computer-assisted-assessment-.}

Računalniško podprti sistemi za vrednotenje znanja.

% To nas toliko zanimalo, razen če kjer najdem podporo za
% učitelje. -\textgreater{}  npr. CodeAcademy možnost sledenja
% napredka učencem in jih vidi učitelj.

\subsection{Prihodnost in smer aplikacij v izobraževanju
  računalništva.}
\label{prihodnost-in-smer-aplikacij-v-izo-rac}

Lahko pogledamo smernice in ugotovimo ali so se zares uresničile? Na
hitrer pogled vse kaže da so se.

Spletna okolja za programiranje * Tabele z kriterijami za ocenjevanje,
zastarelih orodij! (str. 55)

\section{Učenje programiranja}
\label{sec:učenje_programiranja}

\subsection{Zgodovina programskih jezikov v izobraževanju}
\label{sec:zgodovina_programskih_jezikov}

Uporaba računalništva v izobraževanju je bila deležna številnih
sprememb. Sama uporaba računalnika v izobraževanju je tesno
povezana z razvojem računalnikov.  Začetno obdobje, 1960 letih
prejšnjega stoletja so računalniki bili zelo dragi in veliki glavni
računalniki ( \emph{ang. mainframe} ), na njih se je učilo
programiranja, a so se uporabljali tudi za druga področja.
//-\textgreater{} Terminalska obdobje, Poglej gerliča. V tem obdobju
se je za učenje programiranja uporavljal \textbf{FORTRAN} ali
\textbf{asembler}. Programi so bili majhi in enostavi, zaradi fizičnih
omejiteh takratnega delovnega pomnilnika.

V 1970 so na trg prišli manjši računalniki, ki so bili tudi cenejši in
zmogljivejši. V tem času pride v ospredje strukturirano programiranje.
Najpopularnejši programski jezik je bil \textbf{PASCAL}.

V 1980 so se prvič pojavili samostojni osebni računalniki. Programski
jeziki v tem obdobju so bili strukturirani in močnega tipa (
\emph{ang.  strong type.} ). Med te spada \textbf{Ada, Modul 2, ML in
  naj omenimo še Prolog}. V naslednjem desetletji, 1990 so v ospredje
prišli objektno orjentirani programski jeziki, kot sta \textbf{JAVA}
in \textbf{C\#} \cite{thesisAWebP}.

% //Umestitev računalnik v izobraževanju na primarnem področju.
% -\textgreater{} Gerlič Uporaba računalika v namene učenje računalništva
% in programiranja.

% //Doodaj generacije programskih jezikov. //Dodaj vrste programskih
% jezikov.

Metode poučevanja računalništva so se prav tako spreminjale. 1960 so
računalnike uporabljali samo za poučevanje programiranja. Povdarek pri
predmetih programiranja je bil predvsem na detaljlih zmožnosti
programskega jezika. Programiranje je bilo omejeno le na reševanje
enostavnih primerov in povdarek ni bil na reševanju problemov na
splošno.

V 1970 je reševanje problemov in abstrakcija podatkov postala glavni
in najpomembnejši del vseh programerskih predmetov, kar velja še
danes.  Programi so postali večji, bolj interaktivni in spremenil se
je vnos podatkov z tekstovnega v grafičnega. Vsebina predmetov
računalništva se je hitro razširjala, kakor so se množili številni
programski jeziki \cite{thesisAWebP}.

% //Misel: V izobraževanju je potrebno previdno izbrati obseg in področja
% vsebin, pri čemer ne smemo zanemariti timsko delo.

\subsection{Kaj je računalniška znanost?}
\label{sec:kaj_je_računalniška_znanost}

Računalniška znanost ima številne različice definicij, v grobem jo
lahko definicijo strnemo v naslednjih trditvah \cite{guideTCS}.

\begin{itemize}
\item Ukvarja z značilnostmi tisktega kar je izračunljivo.
\item Je znanost, ki izhaja iz več podroćij in ime korenine v matematiki,
znanosti in inženirstvu.
\item ima mnoga podpodročja in je interdisiplinarna z biologijo,
ekonomujo, medicino, zabavo.
\item Ime računalništvo ali računalniška znanost nas lahko tudi zavede in
jo zamenjamo z področjem uporabe računalnika.
\end{itemize}

\subsection{Kaj je programiranje?}
\label{sec:kaj_je_programiranje}

%% Splošne definicije programiranja.
%% Ni le pisanje kode temveč tudi uspešno reševanje nalog in
%% problemov?

\subsection{Programske paradigme}
\label{sec:programske_paradigme}

Paradigma je način kako obravnavamo in gledamo na stavri, je okvir v
katerem leži naša interpretacija realnosti sveta. Paradigma
najpogosteje pomeni vzorec delovanja v znanstvenem ali drugem
raziskovanju.  Izraz -programske paradigme- je več pomenka, ki povzema
mentalne procese, strategije reševanja problemov, povezave med
različnimi paradigmami, programske jezike, stil programiranja in še
več (Wikipedia: Paradigma) \cite{guideTCS}.

Povemo lahko, da je programiranja, hevristična paradigma za algoritme
ki rešujejo probleme. Programski jezik je način za izražanje
programske paradigme.

%//Glavna definicija.
Programske paradigme so hevristike, ki se uporabljajo za reševanje
problemov. Programska paradigma analizira problem, čez specifičen
pogled in na ta način formulira rešitev za dani problem, ki ga razdeli
na manjše dele med katerimi definira razmerja.

Programske paradigme so na primer proceduralno, objektno orientirano,
funkcijsko, logično in istočasno programiranje.

V nadaljevanju bomo spoznali značilnosti dveh programskih paradigem.

\subsection{Proceduralno programiranje}
\label{sec:proceduralno_programiranje}

\subsection{Objektno orientirano programiranje}
\label{sec:objektno_orijentirano_programiranje}



\subsection{Problematika začetkov učenja programiranja}
\label{sec:Problematika_začetkov_učenja_programiranja}

% Po pregledu, ki se ukvarjajo z učenjem programiranja lahko
% ugotovimo, da je samo učenje programiranja tanko staro kot prvi
% program, ki je bil kdaj koli napisan.

% NOTE: Razlikovati moramo med kodiranjem in reševanjem problemov

% Začetne misli o učenju programiranja.
Programiranje je veščina, ki potrebuje veliko vaje, ugotavljajo
avtorji v članku \cite{ITaLCP_DistanceEdu}. Študenti pridobijo znanje
programiranja z veliko programiranja oz. pisanjem kode. Praktični del je
zelo pomemben za proces učenja programiranja.
\cite{ITaLCP_DistanceEdu}.

Nekatere osnove težave, katere srečajo programerji novinci \cite{thesisAWebP}:
\begin{enumerate}
\item
  Inštalacija in nastavitve okolja za programiranje.
\item
  Uporaba urejevalnika besedil.
\item Razumevanje napisanih nalog oz. problemov in uporabe sintakse
  programskega jezika pri pisanju programske kode.
\item
  Razumevanje napak prevajalnika.
\item
  Razhroščevanje.
\end{enumerate}

V preteklosti je bilo razvitih mnogo orodij, ki so nastala ravno z
raziskovanja učenja programiranja, vendar mnoga od teh zahtevajo, da
študenti pišejo celotne programe od začetka do konca.

Tudi začetniki, ki uspešno premagajo začetne ovire in se lotijo
takojšnjega programiranja, imajo zelo slabo napisano in konstruirano
programsko kodo. Pomagati novincem, pistati kvalitetno programsko kodo
je časovno zelo zahtevno opravilo.

%Danes se večini uči OO jezikov, kako podkrepiti to tezo.
Težave programiranja se stopnjujejo ko se za učenje programiranja
uporabljajo Objektno-orjentirani programski jeziki, sej ti zahtevajo
visoko stopnjo abstraktnega razumevanja programskih konceptov in so
načrtovani predvsem za zahtevne programerje.

% Torej je pomembno v katerih programskih jezikih se začnemo učiti
% programiranja? Zakaj sta zato primerna? -> Scratch in Python?  Kje
% je določeno na državnem nivoju da se učit ravno ta dva programska
% jezika

\subsection{Osnovni koncepti programiranja}
\label{sec:Osnvni koncepti_programiranja}

% Predstavljeni koncepti na primerih eden OŠ Scratch drugi Python.
V naslednjem odstavku se bomo vprašali kako lahko formuliramo sintakso
programskega jezika? In kaj je npr. definicija \emph{kopice}.

V ta namen definiramo mehko idejo po avtorju Hazzan \cite{guideTCS},
ki je naslednja. Mehka ideja je koncept, ki mu ne moremo pripisati
toge, niti formalne definicije. Mehke ideje ni niti možno opisati z
točno določeno aplikacijo. Na tem mestu se postavlja vprašanje kako
lahko defineramo nekaj kar se odvija po korakih.

Da odgovorimo na zgornji dve vprašanji, lahko povemo, da so pravila
sintakse togi orisi pri pisanju programske kode in da so semantična
pravila mehke ideje. Opozorimo še na to, da koncepti v računalniški
znanosti niso le toga pravila ali samo mehke ideje, temeč skupek
obojega. V spodnji tabeli \ref{tab:koncept_spremenljivka} prikazuje
primer spremenljivke.


\begin{table}[!htb]

\caption{Prikaz dvojnih, togih in mehkih orisov idej na primeru
  spremenljivke \cite{guideTCS}. }
\label{tab:koncept_spremenljivka}
\begin{tabular}{
  | p{0.30\linewidth-2\tabcolsep} |
  p{0.30\linewidth-2\tabcolsep} |
  p{0.40\linewidth-2\tabcolsep} | }
  \hline
  \rowcolor{gray!50}
  & \textbf{togi orisi} & \textbf{mehki orisi}\\
  \hline
  ime spremenljivke & Pravilo sintakse. & Potreba po imenu
                                          spremenljivke. Katero ime
                                          spremenljivke je pomembno in
                                          zakaj ga je potrebno
                                          določiti.\\
  \hline
  vrednost spremenljivke & Pravila tipa spremenljivke. Rezervacija
                           pomnilnika. & Spremenljivka ima eno
                                         vrednost, ki se lahko
                                         spreminja s časom.\\
  \hline
  dodelitev začetne vrednosti & Pravila sintakse. & Pomen dodelitve
                                                  začetne vrednosti\\
  \hline

\end{tabular}
\end{table}

\subsection{Programiranje v OŠ}
\label{sec:Programiranje_v_OŠ}

\subsection{Programiranje v SŠ}
\label{sec:Programiranje_v_SŠ}


\section{Spletni portali za učenje programiranja}
\label{sec:SPUP}

% NOTE: Zanima nas naslednja vprašanja:
% NOTE: * Kaj so spletni portali za učenje programiranja?
% NOTE: * Zakaj in kje je smiselno uporabljati spletne portale za
% NOTE:   učenje programiranja.
% NOTE: * Prednosti spletnih portalov in slabosti?
% NOTE: * Kako so spletni portali zgrajeni?
% NOTE: * Katere so različne vrste spletnih portalov (Kategorije) in
% NOTE:   katere bodo nas zanimale?
% NOTE: *

V začetku nas bo zanimalo kaj so spletni portali za učenje. Spoznali
bomo, da poznamo različne kategorije spletnih portalov za posredovanje
različnega znanja in veščin. Zanimali nas bodo  predvsem spletni
portali, ki učijo znanje programiranja.

% Kateri tradicionalni spletni portali? Preveri?
% Ali že tu pisati, da v tem primeru gre za učenje na daljavo?!
Tradicionalni spletni portali v izobraževanju, kot so \textbf{moodle},
nikoli niso popolnoma izkoristili zmožnosti uporabe, ki jih ponujajo
nove internetne in komunikacijske tehnologije. Večinoma so se
uporabljale le kot podaljšana roka obstoječim metodam
poučevanja. Uporabljale so se za objavo gradiv in spletno prijavo za
oddajo nalog. Takšni sistemi ne zagotavljajo izboljšav kvalitete
poučevanja programiranja \cite{ITaLCP_DistanceEdu}.

%Posvetili se bomo predvsem takšnim in podobnim spletnim portalom!

Poglejmo primer spletnega portala, ki ga je izdelal avtor
\cite{thesisAWebP}, in ima naslednje elemente.

\begin{enumerate}
\def\labelenumi{\arabic{enumi}.}
\item
  Spletni portal za programiranja, ki omogoča naloge tipa ``Zapolni
  prazna mesta''.
\item
  Ogrodje za analizo, ki preverja kvaliteto in pravilnost, nalog, tipa
  ``Zapolni prazna mesta''.
\item
  Avtomatski sistem za dajanje povratnih informacij, ki sporoča
  prilagojena sporočila prevajalnika in formalni odziv študentom in
  njihovim mentorjem. Poročilo vsebuje kvaliteto napisanega programa,
  strukturo in pravilnost glede na programsko analizo.
\end{enumerate}

\subsection{Predlagane rešitve SPZUP na težave novincev}
\label{predlagane_rešitve_na_težave_novincev}

Pri samem vadenju programiranja je pomembno, da ob težavah, novinci
dobijo čimprajšen odziv mentorja. V velikih razredih se to izkaže za
zelo zahtevno. Z uporabo spletnih tehnologij so v pomoč prav spletni
portali za učenje programiranja. Z njimi lahko razrešimo kar nekaj
tegob, ki jih pestijo novince \cite{thesisAWebP}.

\subsection{Prednosti spletnih portalov za učenje programiranja. }
\label{sec:prednosti_spzup}

Ena od prednosti dela z takšnim sistemom je ta, da novinci niso odvisni
od mentorjevih uradnih govorilnih ur, pravtako tako lahko naloge
opravljajo kadar koli \cite{thesisAWebP}.

%%
%% Pri katerih začetnih težavah nam pomagajo spletni portali!

\subsection{Primer sistemske arhitekture spletnega portala za učenje
  programiranja}
\label{sec:Primer_aritekture_spletnega_portala}

Primer sistemske arhitekture kot so si zamislili avtorji
\cite{ITaLCP_DistanceEdu}. Slika .. opis slike.


\subsubsection{Analiza programske kode}
\label{sec:analiza_programske_kode}

% Različne vrste analiz in povratna informacije o napakah. Kako je to
% dobro urejeno pri spletnih portalih*?

Dober odziv spletnega portala mora dati poročilo o pravilnosti programa in o
kvaliteti \cite{thesisAWebP}.

Ogrodje (ang. framework) za analizo programske kode naj bi vsebovalo:

\begin{itemize}
%\tightlist
\item
  Sintaktično ali semantično opozarjanje na napake ali napake
  kompilerja. //To ima vgrajeno veliko spletnih mest.
\item
  odziv na kvaliteto in pravilnost programske kode //Ali ga sistem nima
  ali je ta pomnanjlkjiv. //Zgornje pomaga predvsem slabpim učencem.
  //Večina sistemov izvaja statično analizo programske kode in tako ni v
  pomoč kakšne kvalitete je ta koda.
\item
  Formalni odzin učitelja oz. komunikacija med učiteljem in učencem.
\end{itemize}

\section{Strategije reševanja problemov}
\label{sec:strategije_reševanja_problemov}

Programiranje je preces pri katerem rešujemo probleme. Reševanje
problemov, zato mora biti središče poučevanja računalniške
znanosti. Reševanje problemov je zahteven mentalni proces. Če na
spletu pobrskamo za strategije reševanja problemov lahko hitro
ugotovimo na obstajajo različne strategije. Kot so recimo naštete na
strani
\href{https://en.wikipedia.org/wiki/Problem_solving#Problem-solving_strategies}{Wikipedia:Reševanje
  problemov (\emph{ang. Problem solving})}, abstrakcija, analogija,
brainstorming, deli in vladaj in mnoge druge.  Proces in tehnike
reševanje problemov se uporablja v mnogih tehničnih in znanstvenih
disciplinah \cite{guideTCS}.

V nekaterih primerih učenci sami razvijejo strategijo s katero rešijo
nek problem. Otroci si na primer sami izmislijo enostvno seštevanje in
odštevanje, dolgo pred tem kadar se to učijo pri pouku
matematike. Toda brez formalne podpore za unčikovito strategijo
reševanja problemov, spodleti še tako inovativnemu učencu tudi pri
enostavnih strategijah kot je \textbf{preizkus in napaka}. Zato je
pomembno, da se uči strategij za reševanje problemov.

\subsection{Proces reševanja problemov}
\label{sec:proces_reševanja_problemov}

Vsak osnoven proces, ki se ukvarja z raševanjem problemov, ne glede na
znanstveno disciplino, se začne z opisom problema. Vsak problem se
navadno zaključi z neko rešitvijo, ki je v nekaterih primerih izražena
z \textbf{zaporedjem korakov} ali \textbf{algoritmom}. V računalništve
algoritem zapišemo z kodo nekega programskega jezika. Zapisan
algoritem testiramo tako, da kodo zaženemo, jo izvedemo. Preden
pridemo od opisa problema do podane rešitve moramo prehoditi kar nekaj
težkih korakov. Na te vmesne korake lahko gledamo kot na procese
odkrivanja, zato lahko na reševanje problemov gledamo tudi kot na
kreativen, umetniški proces \cite{guideTCS}.

Splošno priznani koraki reševanja procesov so naslednji:

\begin{enumerate}
\item \emph{Analiza problema}. Najprej je pomembno da razuemo kaj je
  problem in ga znamo identificirati. Če tega ne znamo, ne moremo
  priti do nobene rešitve.
\item \emph{Alternativnie rešitve}. Razmišljamo o alternativnih
  rešitvah kako bi lahko rešili nek problem.
\item \emph{Izbira pristopa}. Izberemo primeren pristop, kako rešiti problem.
\item \emph{Razgradnja problema}. Problem razgradimo na manjše podprobleme.
\item \emph{Razvoj algoritma}. Algoritem razvijamo po korakih, ki smo
  jih določili v podproblemih.
\item \emph{Pravilnost algoritma}. Preverjanje pravilnosti algoritma.
\item \emph{Učinkovitost algoritma}. Izračunamo učinkovitost algoritma.
\item \emph{Refleksija}. Naredimo refleksijo in analizo na pot, ki smo
  jo naredili pri reševanju problema in naredimo zaključek z tem kar
  lahko izboljšamo za naslednji problem, ki ga bomo reševali.
\end{enumerate}

Točen recept kako se lotiti reševanja ne obstaja. Učencem lahko le
pokažemo nekatere metode in strategije, ki jim lahko pomagajo pri
reševanju problemov. Poglejmo še nekatere pomembne korake podrobneje.

\subsubsection{Razumevaje problema}
\label{sec:razumevanje problema}

Razumevanje problemov je prva stopnja v procesu reševanja
problemov. Pri reševanju algoritemskih nalog najprej moramo
prepoznati, kaj so vhodni podatki in kateri podatki naj bi bili
izhodni. Če znamo povedati kaj bodo vhodni podatki, razumemo tudi
bistvo samega problema.

\subsubsection{Načrtovanje rešitve}
\label{sec:načrtovanje_rešitve}

Novinci se spopadajo z največjimi težavami na začetni stopnji
načrtovanja rešitve za nek problem. V nadaljevanju so predstavljene
tri strategije, ki jih lahko uporabimo na tem koraku reševanja
problema.

\begin{description}
\item [Definicija spremenljivk problema:] Pri rešitvi problema
  si pomagamo tako, da ugotovimo kaj morajo biti vhodni in kateri bojo
  izhodni podatki. S tem razjasnimo problem. V naslednjem koraku
  definiramo \textbf{spremenljivke}, ki so potrebne za rešitev
  problema.
\item [Postopno izboljševanje (\emph{ang. Stepwise
      Refinement)}:]Po tej metodi nas najprej zanima celoten pregled
  strukture problema in odnosi med posameznimi deli. Zatem se šele
  poglobimo specifični in kompleksni implementaciji posameznih pod
  problemov. Postopno izboljševanje je metodologija, ki poteka od
  \textbf{zgoraj-navzdol}, torej od splošnega k specifičnemu. Drugačen
  pristop je od \textbf{spodaj-navzgor}. Za oba pristopa velja da eden
  drugega dopolnjujeta. V obeh primerih je problem razdeljen na manjše
  pod probleme ali naloge. Glavna razlika med obema je mentalni
  proces, ki je potreben za en ali drugi pristop.  V nadaljevanju se
  posvetimo samo pristopu od \textbf{zgoraj-navzdol}. Rešitev, ki jo
  poda \textbf{postopno izboljševanje} ima modularno obliko, ki jo:
  \begin{enumerate}
  \item jo lažje razvijamo in preverjamo,
  \item jo lažje beremo in
  \item nam omogoča, da uporabljamo posamezne pod rešitve tudi za
    reševanje drugih problemov.
  \end{enumerate}
\item [Algoritemski vzorci:] Algoritemski vzorci združujejo
  matematični pogled in elemente načrtovanja. Vzorec podaja načrt na
  rešitev, s katero se srečamo mnogokrat. Algoritemski vzorci so
  primeri elegantnih in učinkovitih rešitev problemov in predstavljajo
  abstraktni model algoritemskega procesa, katerega lahko prilagodimo
  in ga integriramo v rešitve drugim problemom.

  Pri tem procesu lahko nastopi težava prepoznave vzorca algoritma pri
  novincih, saj ti niso sposobni prepoznati podobnosti med posameznimi
  algoritmi ali ne znajo prepoznati bistvo problema, njihove posamezne
  komponente in razmerja med njimi, da bi lahko rešili nove
  probleme. V takih primerih novinci radi ponovno izumijo že njim
  poznane rešitve, ki bi jih lahko uporabili. Te težave navadno
  nastanejo zaradi slabe organizacije sistematike znanja o algoritmih.

  Proces reševanja problemov z algoritemskim vzorcem se navadno začne
  z prepoznavanjem komponent, ki vodijo k rešitvi in iskanjem podobnih
  problemov, na katere še imamo znane rešitve. Zatem prilagodimo
  vzorec prilagodimo za rešitev problema in ga vstavimo v celotno
  rešitev. V večini primerov je potrebno vstaviti več različnih
  vzorcev, da dobimo neko novo rešitev.
\end{description}

\subsubsection{Preverjanje rešitve}
\label{sec:preverjanje_rešitve}
Ko imamo pripravljeno rešitev moramo preveriti ali je ta
pravilna. Pogled na preverjanje pravilnosti rešitve je lahko
teoretične in praktične narave. Razhroščevanje (\emph{ang. debugging})
spada me vrsto aktivnosti, ki nam pomaga pri ugotavljanju pravilnosti
rešitve. Splošno velja da proces razhroščevanja, z programom, ki nam
pomaga razhroščevati (ang. debugger) ali brez njega, poglablja
razumevanje računalniške znanosti. Z tem ko učenci razmišljajo, kako
bodo preverjali ali njihov program deluje pravilno, hkrati v njih
poteka miselni proces refleksije o tem kako so implementirali določen
program in kako ga bojo morebiti morali spremeniti.

Na nivoju do srednje šole uporabljamo praktične metode ugotavljanja
pravilnosti programa, kot je razgroščevanje. Ko želimo znanje
pravilnosti delovanja poglobiti se lahko lotimo tudi teoretične
amalize.

\subsubsection{Refleksija}
\label{sec:refleksije}

Refleksija je mentalni proces ali obnašanje, ki nam omogoča da neko
delovanje analiziramo in o njem tudi premislimo. Refleksija je
pomembno orodje v splošnem učnem procesu, prav tako spadam med
kognitivne procese višjega reda. Z refleksijo učenec dobi priložnost,
da stopi korak nižje in premisli o svojem razmišljanju in tako
izboljša veščino reševanja problemov. Refleksivno razmišljanje je
proces, ki zahteva veliko časa in vaje. Med procesom reševanja
problemov, lahko refleksijo uporabimo na različnih stopnjah.

\begin{itemize}
\item \emph{Pred} reševanjem problemom. Ko problem preberemo, in že
  načrtujemo rešitev, se splača uporabiti refleksijo in razmisliti o
  tem ali smo morda že reševali podoben problem in temu primeren
  vzorec algoritma.
\item \emph{Med} reševanjem problema. Ko rešujemo problem refleksija
  služi, kot pregled, kontrola in nadzor. Na primer, ko nastopijo
  težave pri načrtovanju rešitve ali morda zaznamo težavo ali
  napako. Temo procesu lahko pravimo \textbf{refleksija v akciji}.
\item \emph{Po} reševanju problema. Ko že najdemo rešitev, ki deluje,
  nam refleksija služi kot orodje z katerim pregledamo učinkovitost
  delovanja. Pregledamo strateške odločitve, ki so bile sprejete med
  samim načrtovanjem rešitve.
\end{itemize}

Refleksija je kreativni proces in je pomemben za učenca tako kot za
učitelja.

\section{Metode in strategije pri uporabi spletnih portalov}
\label{sec:Metode_in_strategije_pri_učenju_programiranja}

Primer strategij in metod spletnega portala za učenje \textbf{Jave}.
\cite{thesisAWebP}:

\begin{itemize}
%\tightlist
\item
  Scaffolding -\textgreater{} Gradnja študentovega znanja
  pri katerem pomaga mentorja, z svojim znanjem in izkušnjami.
\item
  Bloomova taskonomija. Zakaj je pomembno vključevanje Bloomove
  taksonomije in kako jo vključujemo.
\item
  Konstruktivizem: Aktivnost študentov pri gradnji znanja. Učenje z
  eksperimentiranjem. Problemski pristop.
\end{itemize}

Kaj od katerih metod predstavlja v uporabi spletnega portala \ldots{}:

\begin{itemize}
%\tightlist
\item
  Spletni portal -\textgreater{} Scaffolding + Bloom
\item
  Naloge narejene tako, da podpirajo konstrutivno metoIn tudi nekatere slabosti, če hih najdem v literaturi
  -\textgreater{} problemski pristopom
\end{itemize}


V naslednjem poglavju sledi pregled tehnik aktivnih metod
poučevanja. V poglavju sledi obravnava didaktičnih pripomočkov, oblik
pouka, in projektno delo \cite{guideTCS}.
\subsection{Didaktični pripomočki}
\label{sec:didaktični_pripomočki}

Med didaktičnimi pripomočki najdemo številna orodja:

\begin{itemize}
\item \textbf{pedagoške igre},
\item računalništvo brez računalništva,
\item \textbf{bogate naloge},
\item miselni vzorci,
\item klasifikacija,
\item metafore.
\end{itemize}

V povezavi z spletnimi portali za učenje programiranja nas bodo
zanimale le nekatera.

\subsubsection{Pedagoške igre}
\label{sec:pedagoške_igre}

\subsubsection{Bogate naloge}
\label{sec:bogate_naloge}

\subsection{Različne oblike pouka}
\label{sec:različne_oblike_pouka}

Računalniško znanost lahko poučujemo tako, da jo predavamo, vendar to
ni v skladu z naprednimi nazori poučevanja aktivnega učenja, ki smo ga
do sedaj spoznali. Za uspešno in koristno učenje se moramo temu
pristopu čim bolj izogniti. To velja predvsem za izobraževanje na
nivoju \textbf{OŠ} in seveda tudi \textbf{SŠ}. Kot smo že poudarili v
poglavju? je pomemben aktivni pristop v učnem procesu.

Pomembno je tudi v kakšni obliki dela poteka pouk. V nasprotju z
frontalnim delom, lahko pouk organiziramo na naslednje načine.

\begin{description}
\item[Samostojno delo:] Prvi način je morda najenostavnejši
  za organizacijo dela v učilnici in omogoča aktivno učenje za vse
  učence. Taka oblika organizacije je primerna predvsem, ko učitelj
  želi preveriti ali vsi učenci sledijo in znajo uporabljati določeno
  znanje in veščine, kot je na primer uporaba integriranega razvojnega
  okolja (\emph{ang. Integrated Development Environment
    (\textbf{IDE})}) ali sledenje določenemu algoritmu.
\item[Delo v parih:] Razred razdelimo v pare, ti rešujejo
  programerske ali ne programerske naloge. V primeru programerskih
  nalog, učenca, ki sta v paru rešujeta programersko nalogo tako da je
  eden v vlogi \textbf{voznika} in drugi v vlogi
  \textbf{navigatorja}. Prvi, voznik ima v nadzoru tipkovnico in
  miško. Drugi sledi razvojnemu procesu in analizira napredek skupaj z
  voznikom. Oba seveda zamenjujeta vloge. Programiranje v parih vodi v
  proces reševanja problemov na dveh nivojih, en nivo predstavlja
  nalogo kodiranja, drugi predstavlja uporabo strategij pri reševanju
  problema. Ko so naloge niso programerske in jih ne izvajamo na
  računalniku, zgubimo nalogo voznika. Kljub temu lahko izvajamo tako
  obliko pouka, saj lahko sklepamo, da je delo v parih, pri reševanju
  problemov, primernejše kot v večjih skupinah, kjer obstaja večja
  možnost, da nekateri učenci dominirajo v skupini in teko druge
  učence prikrajšajo za sodelovanje.
\item[Skupinsko delo] Druga oblika organizacije je delo v
  skupi ali timu in je primerna v naslednjih primerih:
  \begin{enumerate}[a.]
  \item ko sta potrebna več kot dva učenca za opravilo neke naloge,
  \item Ko učitelj želi izkoristiti raznolikost v skupini,
  \item ko je razred razmeroma velik in si učitelj želi olajšati delo
    tako da učence razdeli v manjše skupine,
  \item ko želi da so vključeni vsi učenci, a le eden iz posamezne
    skupine naj bi predstavljal narejeno delo.
  \end{enumerate}
\item[Skupinsko delo - sestavljanka (\emph{ang. Jigsaw Classroom})] Po
  navodilih spletne strani \href{https://www.jigsaw.org/}{razredne
    sestavljanke (\emph{ang. Jigsaw classroom})} je oblika
  organizacije na naslednji način.
  \begin{enumerate}
  \item Učence razdelimo na skupine 5 - 6. Vsak od njih ima nalogo, da
    predela posamezno nalogo, poglavje, ki je razdeljeno na toliko pod
    poglavij ko je učencev v skupini. Vsak od njih je odgovoren, da se
    nauči posamezno podpoglavje in to znanje posreduje naprej drugim
    učencem.
  \item Preden učenci preidejo k poročanju posameznega podpoglavja, se
    sestanejo z učenci drugih skupin, ki imajo isto nalogo oz
    podpoglavje. S tem zagotovimo večjo točnost naučenega.
  \item Učenci se vrnejo nazaj v svoje prvotne heterogene skupine, in
    poučijo svoje sošolce o tem kaj so se naučili.
  \end{enumerate}
  Učitelj se odloči kaj po končni izdelek, ali bo to napisano kratko
  poročilo, ali plakat, ali kak drugi pisni izdelek, delo se lahko
  zaključi tudi brez končnega izdelka.

  Kot je razvidno z organizacije dela \textbf{sestavljanke} so
  prednosti ogromne, tiste ki omogočajo kognitivni razvoj in tiste, ki
  socialnega. Učenje v tej obliki vzpodbuja učenje, poslušanje,
  sodelovanje in deljenje znanja.
\end{description}

\subsection{Projektno delo}
\label{sec:projektno_delo}

Preglejmo najprej nekatere lastnosti, ki jih prinaša projektno
delo. To lahko poteka tako, da učenci delajo samostojno ali v
skupini. Učitelj je tisti, ki vodi proces projektnega dela. Učenec je
pri projektnem delu bistveni člen in mu tako omogoča aktivno učenje.

\subsection{Učne strategije}
\label{sec:učne_strategije}

\subsubsection{Aktivno učenje in model aktivnega učenja}
\label{sec:aktivno_učenje_in_model_aktivnega_učenja}

Vsak pouk računalništva mora biti zgrajen kot model in bi moral
upoštevati naslednja načela:
\begin{itemize}
\item Vzpodbujati mora študente z pozitivno naravnanim poukum in
   omogočati mora okolje kjer študent najde pomoč.
 \item Pouk računalništva je grajen na konstruktivnih metodah poučevanja
   in aktivnem učenju.
\end{itemize}

\textbf{Konstruktivizem} je kognitivna teorija, ki preučuje naravo
procesov učenja. Po tem principu naj bi učenci konstruirali novo
znanje na osnovi preurejanja in izpopolnjevanja že obstoječega
znanja. Znanje se gradi na obstoječih mentalnih strukturah in na
odzivu, ki ga dobi učenec iz učnega okolja. Mentalne strukture so
grajene korak za korakom, ena za drugo, seveda s to metodo lahko pride
tudi do sestopanja ali slepih koncev. Proces je povezan z Piagetovim
mehanizmom asimilacije \cite{guideTCS}.

Pri \textbf{aktivnem učenju} je najpomembnejše to, da učenci z lastno
aktivnostjo ugotovijo, sami za sebe kako nekaj deluje. Sami si morajo
izmisliti primere, preiskusiti lastne veščine in reševati neloge, ki
so jih že ali jih še podo spoznali. Učenje je aktivno usvajanje, je
gradnja idej in znanja. Za učenje mora biti posameznik aktivno
vključen v gradnjo svojih lastnih mentalnih modelov.

Model aktivnega učenja je sestavljen s štirih korakov \cite{guideTCS}.

\begin{itemize}
\item \textbf{Sprožilec} Je je naloga, ki predstavlja  iziv za uvod v novo
tematiko.  //Gerlič -> Motivacija.
\item \textbf{aktivnost} Študenti izvajajo aktivnost, ki jim je bila
predstavljena v sprožilcu. Ta kora je lahko kratek ali lahko
zavzame večju del učne ure. To je odviso od vrste sprožilca in
izobraževalnih ciljev.
\item \textbf{diskusija} sledi po koncu aktivnosti, kjer se zbere zeloten
razred, neglede na obliko dela. V temo koraku študenti izpopolnijo
koncepte in ideje, kod del konstruktivnega učnega procesa.
\item \textbf{povzetek} je lahko izračen v različnih oblikah, kot so
zaogrožene definicije, lahko so miselni vzorci ali povezav med
temami, ki so jih obravnavali študenti in med drugimi temami, ki se
navezujejo nanje.
\end{itemize}

Ko se ta model izkaže za primernega, ga lahko uporabimo v številnih
učnih urah v različnih variacijah.

\subsubsection{Učenje  na daljavo}
\label{sec:Učenje_na_daljavo}


\subsection{Tipi nalog}
\label{tipi_nalog}

\subsubsection{Zapolni prazna mesta}
\label{sec:zapolni_prazna_mesta}

Tip nalog začenniku ponuja ogrodje programa, del programske kode, na
katerem dijak usvoji novo znanje in/ali lahko uporablja že pridobljeno
znanje.


\section{Kategoriziranje spletnih portalov}
\label{sec:kategoriziranje_spletnih_portalov}

\subsection{Vrsta vsebine}
\label{sec:Razvrstitev_spletnih_portalov}

Po hitrem pregledu izbranih spletnih portalov lahko ugotovimo, da je

\subsection{Programski jeziki}
\label{sec:programski_jeziki}


\section{Ovrednotenje izbranih spletnih portalov in njihove posebnosti}
\label{sec:pregled_spletnih_portalov}

\subsection{Pogoji za ožji izbor spletnih portalov}
\label{sec:pogoji_za_ožji_izbor_sp}



\subsection{Določitev Kriterijev}
\label{sec:dolocitev_kriterijev}

Pri vrednotenju spletnih portalov bomo upoštevali naslednje
kriterije,

%Zaenkrat samo povzeto, to kar sem spoznal do sedaj.

\begin{itemize}
\item število programskih jezikov,
\item zahtevano predznanje uporabnika,
\item interaktivna povratna informacija,
\item problemski pristop,
\item jezik spletne strani.
\end{itemize}

\section{Možni načini uporabe spletnih portalov pri pouku}
\label{sec:načini_uporabe_sp}

% \newpage
% %%Recept za vstavljanje poljubnega naslova za Viri in literatura
\cleardoublepage
% \renewcommand*{\refname}{Literatura in viri}
% %\renewcommand*{\refname}{\vspace*{-12mm}}
% %\section*{Viri in Literatura}
\phantomsection
\addcontentsline{toc}{section}{Literatura in viri}
% %\addcontentsline{toc}{section}{Literatura}
\begin{thebibliography}{9}

% NOTE: Citiranje sem povzel po ieee z dokumenta ieeecitationref.pdf.
% NOTE: Spodaj dodajam
%
%nekaj uporabljenih primerov člankov, ki jih
% NOTE: bom potrebovalmm

% TODO: Prosi za pomoč pri oblikovanju in uporabi literature.

\bibitem{model_uporabe_rac_izo-web}
  Martina Fefer,
  \emph{Uporaba informacijske-komunikacijske  tehnologije v osnovnih
    šolah s prilagojenim programom},
  Univerza v Mariboru - Fakulteta za naravoslovje in matematiko, Maribor,
  1999. Pridobljeno 4.4. 2016, iz \url{http://student.pfmb.uni-mb.si/~dgunze/diplomske/d2/s6.html}.

\bibitem{LTProg01}
  Anthony Robins, Janet Rountree, and Nathan Rountree,
  ''Learning and Teaching Programming: A Review and
  Discussion'' v \emph{Comuper Science education}, vol 13, No. 2,
  2003, pp. 137 - 172.
  % Univerza: Computer Science, University of Otago, Dunedin, New Zealand.

\bibitem{ITaLCP_DistanceEdu}
  S.C. Ng, S.O Choy, R. Kwan, S.F. Chan,
  ''A Web-Based Environment to Improve Teaching and Learning of
  Computer Programming in Distance Education'', \emph{ICWL'05
    Proceedings of the 4th international conference on Advances in
    Web-Based Learning}, 2005

\bibitem{mentalModels}
  - L. Ma, J. D. Ferguson, M. Roper, I.Ross, M. Wood,
  ''A web-based learning model for improving programming students' mental
  models'', v \emph{Proceedings of the 9th annual conference of the subject
  centre for information and computer sciences}, HE Academy, 2008
  pp. 88-94.

\bibitem{guideTCS}
  O. Hazzan, T. Lapidot, N. Ragonis,
  \emph{Guide to Teaching Computer Science}, Springer, 2011.

\bibitem{thesisAWebP}
  Nghi Truong,
  \emph{A web-based programming environment for novice programmers},
  Queensland University of Technology, Australia, 2007.



% \bibitem{bilten01}
%   \emph{Tematska številka Biltena E-šolstvo}, \\
%   Bilten E-šolstva, Številka: 2010/1\\
%  elektronski vir: \url{http://projekt.sio.si/wp-content/uploads/sites/8/2015/01/E-solstvo_BILTEN_2010-1_screen.pd}\\
% (E-središče v okviru projekta E-šolstvo, 2010)

% \bibitem{test}
%  CPI, \emph{O CPI}, \\
%   \url{http://www.cpi.si/o-cpi.aspx},\\
%   Obiskano: \emph{\datum}

%  \bibitem{esolska_torba}
%   SIO, \emph{E-šolska torba}, \\
%    \url{http://projekt.sio.si/e-solska-torba/},\\
%    Obiskano: \emph{\datum}

%  \bibitem{sio_arhiv}
%   SIO, \emph{Repozitorij gradiv}, \\
%    \url{http://portal.sio.si/no_cache/sio/gradiva/repozitorij_gradiv_trubar/},\\
%    Obiskano: \emph{\datum}

%  \bibitem{stran_execute}
%   \emph{Spletna stran execute}, \\
%    \url{http://execute.fnm.uni-mb.si},\\
%    Obiskano: \emph{\datum}


% \bibitem{iUčbeniki}
%   Andreja Čuk ... [et al.], \\
%   \emph{SLOVENSKI i-učbeniki} [elektronski vir], \\
%   način dostopa: \url{http://www.zrss.si/pdf/slovenski-i-ucbeniki.pdf}\\
%   (Zavod republike slovenije za šolstvo, 2014)


% \bibitem{dipl_simulacije1}
%   Vera Kožuh \\
%   \emph{Diplomska seminarska naloga - Simulacije z računalnikom pri
%     pouku fizike v osnovni šoli}, \\
%  elektronski vir: \url{http://splet-stari.fnm.uni-mb.si/pefmb_old/didgradiva/diplome/kozuh/}\\
% (Pedagoška fakulteta - Oddelek za fiziko, Maribor 1999)


% \bibitem{informatika_2010}
%   Avtor, \\
%   \emph{Naslov} \\
% (Založba, Mesto 2006)



\end{thebibliography}


  %%% Local Variables:
  %%% mode: latex
  %%% TeX-master: "diploma"
  %%% End:



%%% End document
\end{document}


%%% Local Variables:
%%% mode: latex
%%% TeX-master: t
%%% End:
