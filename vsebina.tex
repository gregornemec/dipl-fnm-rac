%% UVOD je posebej datoteka zaradi lažjega upravljanja.


\section{Uporaba računalnika v izobraževanju}
\label{sec:uporaba-raunalnika-v}

Po modelu uporabe računalnika v izobraževanju, njegova uporaba spada v
primarno področje, saj sem prištevamo aktivnosti s katerimi želimo
uporabnike seznaniti z delovanjem in uporabo računalnika oz. sodobno
informacijsko komunikacijsko tehnologijo
(\textbf{IKT})\cite{model_uporabe_rac_izo-web}.


\subsection{Zgodovin uporabe računalnika v izobraževanju
računalništva.}\label{zgodovina_uporabe_racunalnika_v_izobrazevanju}

Zaradi predstaveh sprememb, ki jih je bila uporaba računalnika v
izobraževanju deležna preglejmo na kratko zgodovino. Zama uporaba
računalnika v izobraževanju je tesno povezana z razvojem računalnikov.
Začetno obdoje, 1960 letih prejšnjega stoletja so računalniki bili zelo
dragi in veliki glavni računalniki ( \emph{ang. mainframe} ), na njih se
je učilo programiranja, a so se uporabljaji tudi za druga področja.
//-\textgreater{} Terminalska obdobje, Poglej gerliča. V tem obdobju se
je za učenje programiranja uporavljal \textbf{FORTRAN} ali
\textbf{asembler}. Programi so bili majhi in enostavi, zaradi fizičnih
omejiteh takratnega delovnega pomnilnika.

V 1970 so na trg prišli manjši računalniki, ki so bili tudi cenejši in
zmoglivejši. V tem času pride v ospredje strukturirano programiranje.
Najpopularnejši programski jezik je bil \textbf{PASCAL}.

V 1980 so se privič pojavili samostojni osebni računalniki. Programski
jeziki v tem obdobju so bili strukturirani in močnega tipa ( \emph{ang.
strong type.} ). Med te spada \textbf{Ada, Modul 2, ML in naj omenimo še
Prolog}. V naslednjem desetletji, 1990 so v ospredje prišli objektno
orjentirani programski jeziki, kot sta \textbf{JAVA} in \textbf{C\#}.

% //Umestitev računalnik v izobraževanju na primarnem področju.
% -\textgreater{} Gerlič Uporaba računalika v namene učenje računalništva
% in programiranja.

% //Doodaj generacije programskih jezikov. //Dodaj vrste programskih
% jezikov.

Metode poučevanja računalništva so se pravtako spreminjale. 1960 so
računalnike uporabljali samo za poučevanje programiranja. Povdarek pri
predmetih programiranja je bil predvsem na detaljlih zmožnosti
programskega jezika. Programiranje je bilo omejeno le na reševanje
enostavnih primerov in povdarek ni bil na reševanju problemov na
splošno.

V 1970 je reševanje problemov in abstrakcija podatkov postala glavni in
najpomembnejši del vseh programerskih predmetov, kar velja še danes.
Programi so postali večji, bolj interaktivni in spremenil se je vnos
podatkov z tekstovnega v grafičnega. Vsebina predmetov računalništva se
je hitro razširjala, kakor so se množili številni programski jeziki.

% //Misel: V izobraževanju je potrebno previdno izbrati obseg in področja
% vsebin, pri čemer ne smemo zanemariti timsko delo.

\section{Problematika začetkov učenja programiranja}
\label{sec:Problematika_začetkov_učenja_programiranja}

% Po pregledu člankov, ki se ukvarjajo z učenjem programiranja lahko
% ugotovimo, da je samo učenje programiranja tanko staro kot prvi
% program, ki je bil kdaj koli napisan.

% NOTE: Razlikovati moramo med kodiranjem in reševanjem problemov

% Začetne misli o učenju programiranja.
Programiranje je veščina, ki potrebuje veliko vaje, ugotavljajo
avtorji v članku \cite{ITaLCP_DistanceEdu}. Študenti pridobijo znanje
programiranja z veliko programiranja, pisanjem kode. Praktični del je
zelo pomemben za proces učenja programiranja.
\cite{ITaLCP_DistanceEdu}.

Nekatere osnove težave, katere srečajo programerji novinci \cite{thesisAWebP}:
\begin{enumerate}
\item
  Inštalacija in nastavitve okolja za programiranje.
\item
  Uporaba urejevalnika besedil.
\item
  Razumevanje programskih vprašanj in uporabe sintakse programskega
  jezika pri pisanju programske kode.
\item
  Razumevanje napak prevajalnika.
\item
  Razhroščevanje.
\end{enumerate}

Težave programiranja se stopnjujejo ko se za učenje progremiranja
uporabljajo Objektno-orjentirani programski jeziki, sej ti zahtevajo
visoko stopnjo abstraktnega razumevanja programskih konceptov in so
načrtovani predvsem za zahtevne programerje.

% Torej je pomembno v katerih programskih jezikih se začnemo učiti
% programiranja? Zakaj sta zato primerna? -> Scratch in Python?  Kje
% je določeno na državnem nivoju da se učit ravno ta dva programska
% jezika


\section{Spletni portali za učenje programiranja}
\label{sec:SPUP}

% NOTE: Zanima nas naslednja vprašanja:
% NOTE: * Kaj so spletni portali za učenje programiranja?
% NOTE: * Zakaj in kje je smiselno uporabljati spletne portale za
% NOTE:   učenje programiranja.
% NOTE: * Prednosti spletnih portalov in slabosti?
% NOTE: * Kako so spletni portali zgrajeni?
% NOTE: * Katere so različne vrste spletnih portalov (Kategorije) in
% NOTE:   katere bodo nas zanimale?
% NOTE: *

V začetku nas bo zanimalo kaj so spletni portali za učenje. Spoznali
bomo, da poznamo različne kategorije spletnih portalov za posredovanje
različnega znanja in veščin. Zanimali nas bodo  predvsem spletni
portali, ki učijo znanje programiranja.

% Kateri tradicionalni spletni portali? Preveri?
% Ali že tu pisati, da v tem primeru gre za učenje na daljavo?!
Tradicionalni spletni portali v izobraževanju, kot so \textbf{moodle},
nikoli niso popolnoma izkoristili zmožnosti uporabe, ki jih ponujajo
nove internetne in komunikacijske tehnologije. Večinoma so se
uporabljale le kot podaljšana roka obstoječim metodam
poučevanja. Uporabljale so se za objavo gradiv in spletno prijavo za
oddajo nalog. Takšni sistemi ne zagotavljajo izboljšav kvalitete
poučevanja programiranja \cite{ITaLCP_DistanceEdu}.

%% Pri katerih začetnih težavah nam pomagajo spletni portali!


\subsection{Primer sistemske arhitekture spletnega portala za učenje
  programiranja}
\label{sec:Primer_aritekture_spletnega_portala}




\section{Osnovni koncepti programiranja}
\label{sec:Osnvni koncepti_programiranja}

\section{Kriteriji ocenjevanja spletnih portalov}
\label{sec:Kriteriji_ocenjevanja}

\subsection{Razvrstitev spletnih portalov glede na ponujeno vrsto
  vsebine}
\label{sec:Razvrstitev_spletnih_portalov}





\section{Metode in strategije pri uporabi spletnih portalov}
\label{sec:Metode_in_strategije_pri_učenju_programiranja}

\subsection{Učenje  na daljavo}
\label{sec:Učenje_na_daljavo}

Pri učenju


\section{Programiranje v OŠ}
\label{sec:Programiranje_v_OŠ}

\section{Programiranje v SŠ}
\label{sec:Programiranje_v_SŠ}




%%% Local Variables:
%%% mode: latex
%%% TeX-master: "diploma"
%%% End:
