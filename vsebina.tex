%% UVOD je posebej datoteka zaradi lažjega upravljanja.




\section{Uporaba računalnika v izobraževanju}
\label{sec:uporaba-raunalnika-v}

Po modelu uporabe računalnika v izobraževanju, njegova uporaba pri
učenju programiranja spada v primarno področje, saj sem prištevamo
aktivnosti s katerimi želimo uporabnike seznaniti z delovanjem in
uporabo računalnika oz. sodobno informacijsko komunikacijsko
tehnologijo (\textbf{IKT}) \cite{model_uporabe_rac_izo-web}. Računalnik
je tista učna vsebina, ki jo obravnavamo. Računalništvo v osnovi
nastopa v dveh pomembnih področjih \cite{gerlic_2000}:

\begin{itemize}
\item kot element splošne izobrazbe,
\item kot element ožje strokovne - poklicne izobrazbe
  oz. usposabljanja.
\end{itemize}

V današnjem času se računalnik kot element splošne izobrazbe kaže kot
velika potreba oz. se zdi znanje njegove uporabe samoumevno. Že pri
najmlajših otrocih računalnik vzbuja zanimanje in interes.  Računalnik
je postal intelektualno orodje in pripomoček v vsaki sferi človekove
dejavnosti in je prodrl tudi v šolo. Tako imenovana \emph{računalniška
  pismenost} postaja nuja in zajema vse to kar bi človek moral znati o
računalniku in to, kako je potrebno z njim delati, da bo uspešno živel
v družbi, ki je osnovana na informacijah (informacijski družbi)
\cite{klemencic_2011}.

%% Kje je programiranje, ali spada v strokovno izobraževanje in ne
%% spada v splošno računalniško pismenost, ali je danes zahteva po
%% zanju programiranju le to že uvrstila med splošno pismenost.


%%Podrobneje kje se izvaja računalniško izobraževanje v OŠ in SŠ
Računalništvo se ne izvaja kot redni predmet, temveč poteka v


\subsection{Zgodovin uporabe računalnika v izobraževanju
računalništva.}\label{zgodovina_uporabe_racunalnika_v_izobrazevanju}

%Splošna zgodovina uporabe računalništva v izobraževanju

\subsection{Pregled aplikacij v izobraževanju
računalništva.}\label{pregled-aplikavij-v-izobraux17eevanju-raux10dunalniux161tva.}

V katero kategorijo spadajo spletne strani za učenje programiranja?
Klasifikacija!?

\subsubsection{\texorpdfstring{Sisatemi CAI ( \emph{ang Computer
Assisted Instruction}
)}{Sisatemi CAI ( ang Computer Assisted Instruction )}}\label{sisatemi-cai-ang-computer-assisted-instruction}

Računalniško podprt sistem za učenje. -\textgreater{} Točna definicija
gerlič.

Katere stvari mora nujno posedovati orodje za učenje novincem, da jim je
kar se da v največjo pomoč?

Pomembna sta dva glavna cilja na katerih temelji učenje programerjev
novincev:

\begin{enumerate}
\def\labelenumi{\arabic{enumi}.}
%\tightlist
\item
  Sistemi za učenje, pomagajo izključno učenju programiranja.
\item
  Sistemi za pomoč pri programiranju, ko želimo programiranje uporabiti
  za doego nekega drugega cilja.
\end{enumerate}

\begin{itemize}
%\tightlist
\item
  Novice programming system and languge taxonomy - tabela (str. 43).
\end{itemize}

Skupine programskih orodij, ki so v pomoč novincem so naslednje:

\begin{enumerate}
\def\labelenumi{\arabic{enumi}.}
%\tightlist
\item
  mikro svetovi,
\item
  vizualna okolja za programiranje,
\item
  okolje za izdelavo modela poteka,
\item
  okolje za izdelovanje objektov,
\item
  okolje za risanje in realizacijo algoritmov.
\end{enumerate}

\subsubsection{\texorpdfstring{Sistemi CAA ( \emph{ang. Computer
Assisted assessment}
).}{Sistemi CAA ( ang. Computer Assisted assessment ).}}\label{sistemi-caa-ang.-computer-assisted-assessment-.}

Računalniško podprti sistemi za vrednotenje znanja. // To nas toliko
nebo zanimalo, razen če kjer najdem podporo za učitelje. -\textgreater{}
Ima npr. CodeAcademy možnost sledenja napredka učencem in jih vidi
učitelj.

\subsection{Prihodnost in smer aplikacij v izobraževanju
računalništva.}\label{prihodnost-in-smer-aplikacij-v-izobraux17eevanju-raux10dunalniux161tva.}

Lahko pogledamo smernice in ugotovimo ali so se zares uresničile? Na
hitrer pogled vse kaže da so se.

Spletna okolja za programiranje * Tabele z kriterijami za ocenjevanje,
zastarelih orodij! (str. 55)

\section{Učenje programiranja}
\label{sec:učenje_programiranja}

\subsection{Zgodovina programskih jezikov v izobraževanju}
\label{sec:zgodovina_programskih_jezikov}

Uporaba računalništva v izobraževanju je bila deležna številnih
sprememb. Sama uporaba računalnika v izobraževanju je tesno
povezana z razvojem računalnikov.  Začetno obdobje, 1960 letih
prejšnjega stoletja so računalniki bili zelo dragi in veliki glavni
računalniki ( \emph{ang. mainframe} ), na njih se je učilo
programiranja, a so se uporabljali tudi za druga področja.
//-\textgreater{} Terminalska obdobje, Poglej gerliča. V tem obdobju
se je za učenje programiranja uporavljal \textbf{FORTRAN} ali
\textbf{asembler}. Programi so bili majhi in enostavi, zaradi fizičnih
omejiteh takratnega delovnega pomnilnika.

V 1970 so na trg prišli manjši računalniki, ki so bili tudi cenejši in
zmoglivejši. V tem času pride v ospredje strukturirano programiranje.
Najpopularnejši programski jezik je bil \textbf{PASCAL}.

V 1980 so se privič pojavili samostojni osebni računalniki. Programski
jeziki v tem obdobju so bili strukturirani in močnega tipa (
\emph{ang.  strong type.} ). Med te spada \textbf{Ada, Modul 2, ML in
  naj omenimo še Prolog}. V naslednjem desetletji, 1990 so v ospredje
prišli objektno orjentirani programski jeziki, kot sta \textbf{JAVA}
in \textbf{C\#} \cite{thesisAWebP}.

% //Umestitev računalnik v izobraževanju na primarnem področju.
% -\textgreater{} Gerlič Uporaba računalika v namene učenje računalništva
% in programiranja.

% //Doodaj generacije programskih jezikov. //Dodaj vrste programskih
% jezikov.

Metode poučevanja računalništva so se pravtako spreminjale. 1960 so
računalnike uporabljali samo za poučevanje programiranja. Povdarek pri
predmetih programiranja je bil predvsem na detaljlih zmožnosti
programskega jezika. Programiranje je bilo omejeno le na reševanje
enostavnih primerov in povdarek ni bil na reševanju problemov na
splošno.

V 1970 je reševanje problemov in abstrakcija podatkov postala glavni
in najpomembnejši del vseh programerskih predmetov, kar velja še
danes.  Programi so postali večji, bolj interaktivni in spremenil se
je vnos podatkov z tekstovnega v grafičnega. Vsebina predmetov
računalništva se je hitro razširjala, kakor so se množili številni
programski jeziki \cite{thesisAWebP}.

% //Misel: V izobraževanju je potrebno previdno izbrati obseg in področja
% vsebin, pri čemer ne smemo zanemariti timsko delo.

\subsection{Kaj je programiranje?}
\label{sec:kaj_je_programiranje}

%% Splošne definicije programiranja.
%% Ni le pisanje kode temveč tudi uspešno reševanje nalog in
%% problemov?

\subsection{Programske paradigme}
\label{sec:programske_paradigme}

Paradigma je način kako obravnavamo in gledamo na stavri, je okvir v
katerem leži naša interpretacija realnosti sveta. Paradigma
najpogosteje pomeni vzorec delovanja v znanstvenem ali drugem
raziskovanju.  Izraz -programske paradigme- je več pomenka, ki povzema
mentalne procese, strategije reševanja problemov, povezave med
različnimi paradigmami, programske jezike, stil programiranja in še
več (Wikipedia: Paradigma).

Povemo lahko, da je programiranja, hevristična paradigma za algoritme
ki rešujejo probleme. Programski jezik je način za izražanje
programske paradigme.

%//Glavna definicija.
Programske paradigme so hevristike, ki se uporabljajo za reševanje
problemov. Programska paradigma analizira problem, čez specifičen
pogled in na ta način formulira rešitev za dani problem, ki ga razdeli
na manjše dele med katerimi definira razmerja.

Programske paradigme so na primer --proceduralno, objektno orientirano,
funkcijsko, logično in istočasno programiranje--.

\subsection{Problematika začetkov učenja programiranja}
\label{sec:Problematika_začetkov_učenja_programiranja}

% Po pregledu, ki se ukvarjajo z učenjem programiranja lahko
% ugotovimo, da je samo učenje programiranja tanko staro kot prvi
% program, ki je bil kdaj koli napisan.

% NOTE: Razlikovati moramo med kodiranjem in reševanjem problemov

% Začetne misli o učenju programiranja.
Programiranje je veščina, ki potrebuje veliko vaje, ugotavljajo
avtorji v članku \cite{ITaLCP_DistanceEdu}. Študenti pridobijo znanje
programiranja z veliko programiranja oz. pisanjem kode. Praktični del je
zelo pomemben za proces učenja programiranja.
\cite{ITaLCP_DistanceEdu}.

Nekatere osnove težave, katere srečajo programerji novinci \cite{thesisAWebP}:
\begin{enumerate}
\item
  Inštalacija in nastavitve okolja za programiranje.
\item
  Uporaba urejevalnika besedil.
\item Razumevanje napisanih nalog oz. problemov in uporabe sintakse
  programskega jezika pri pisanju programske kode.
\item
  Razumevanje napak prevajalnika.
\item
  Razhroščevanje.
\end{enumerate}

V preteklosti je bilo razvitih mnogo orodij, ki so nastala ravno z
raziskovanja učenja programiranja, vendar mnoga od teh zahtevajo, da
študenti pišejo celotne programe od začetka do konca.

Tudi začetniki, ki uspešno premagajo začetne ovire in se lotijo
takojšnjega programiranja, imajo zelo slabo napisano in konstruirano
programsko kodo. Pomagati novincem, pistati kvalitetno programsko kodo
je časovno zelo zahtevno opravilo.

Težave programiranja se stopnjujejo ko se za učenje progremiranja
uporabljajo Objektno-orjentirani programski jeziki, sej ti zahtevajo
visoko stopnjo abstraktnega razumevanja programskih konceptov in so
načrtovani predvsem za zahtevne programerje.

% Torej je pomembno v katerih programskih jezikih se začnemo učiti
% programiranja? Zakaj sta zato primerna? -> Scratch in Python?  Kje
% je določeno na državnem nivoju da se učit ravno ta dva programska
% jezika

\subsection{Predlagane rešitve}
\label{sec:predlagane_resitve}

Pri samem vadenju programiranja je pomembno, da ob težavah, novinci
dobijo čimprajšen odziv mentorja. V velikih razredih se to izkaže za
zelo zahtevno.


\subsection{Osnovni koncepti programiranja}
\label{sec:Osnvni koncepti_programiranja}

\subsection{Programiranje v OŠ}
\label{sec:Programiranje_v_OŠ}

\subsection{Programiranje v SŠ}
\label{sec:Programiranje_v_SŠ}


\section{Spletni portali za učenje programiranja}
\label{sec:SPUP}

% NOTE: Zanima nas naslednja vprašanja:
% NOTE: * Kaj so spletni portali za učenje programiranja?
% NOTE: * Zakaj in kje je smiselno uporabljati spletne portale za
% NOTE:   učenje programiranja.
% NOTE: * Prednosti spletnih portalov in slabosti?
% NOTE: * Kako so spletni portali zgrajeni?
% NOTE: * Katere so različne vrste spletnih portalov (Kategorije) in
% NOTE:   katere bodo nas zanimale?
% NOTE: *

V začetku nas bo zanimalo kaj so spletni portali za učenje. Spoznali
bomo, da poznamo različne kategorije spletnih portalov za posredovanje
različnega znanja in veščin. Zanimali nas bodo  predvsem spletni
portali, ki učijo znanje programiranja.

% Kateri tradicionalni spletni portali? Preveri?
% Ali že tu pisati, da v tem primeru gre za učenje na daljavo?!
Tradicionalni spletni portali v izobraževanju, kot so \textbf{moodle},
nikoli niso popolnoma izkoristili zmožnosti uporabe, ki jih ponujajo
nove internetne in komunikacijske tehnologije. Večinoma so se
uporabljale le kot podaljšana roka obstoječim metodam
poučevanja. Uporabljale so se za objavo gradiv in spletno prijavo za
oddajo nalog. Takšni sistemi ne zagotavljajo izboljšav kvalitete
poučevanja programiranja \cite{ITaLCP_DistanceEdu}.

Poglejmo primer spletnega portala, ki ga je izdelal avtor
\cite{thesisAWebP}, in ima naslednje elemente.

\begin{enumerate}
\def\labelenumi{\arabic{enumi}.}
\item
  Spletni portal za programiranja, ki omogoča naloge tipa ``Zapolni
  prazna mesta''.
\item
  Ogrodje za analizo, ki preverja kvaliteto in pravilnost, nalog, tipa
  ``Zapolni prazna mesta''.
\item
  Avtomatski sistem za dajanje povratnih informacij, ki sporoča
  prilagojena sporočila prevajalnika in formalni odziv študentom in
  njihovim mentorjem. Poročilo vsebuje kvaliteto napisanega programa,
  strukturo in pravilnost glede na programsko analizo.
\end{enumerate}

\subsection{Prednosti spletnih portalov za učenje programiranja. }
\label{sec:prednosti_spzup}

Ena od prednosti dela z takšnim sistemom je ta, da novinci niso odvisni
od mentorjevih uradnih govorilnih ur, pravtako tako lahko naloge
opravljajo kadar koli \cite{thesisAWebP}.

%% Pri katerih začetnih težavah nam pomagajo spletni portali!

\subsection{Primer sistemske arhitekture spletnega portala za učenje
  programiranja}
\label{sec:Primer_aritekture_spletnega_portala}

Primer sistemske arhitekture kot so si zamislili avtorji
\cite{ITaLCP_DistanceEdu}. Slika .. opis slike.

\subsubsection{Analiza programske kode}
\label{sec:analiza_programske_kode}

% Različne vrste analiz in povratna informacije o napakah. Kako je to
% dobro urejeno pri spletnih portalih*?

Dober odziv spletnega portala mora dati poročilo o pravilnosti programa in o
kvaliteti \cite{thesisAWebP}.

Ogrodje (ang. framework) za analizo programske kode naj bi vsebovalo:

\begin{itemize}
%\tightlist
\item
  Sintaktično ali semantično opozarjanje na napake ali napake
  kompilerja. //To ima vgrajeno veliko spletnih mest.
\item
  odziv na kvaliteto in pravilnost programske kode //Ali ga sistem nima
  ali je ta pomnanjlkjiv. //Zgornje pomaga predvsem slabpim učencem.
  //Večina sistemov izvaja statično analizo programske kode in tako ni v
  pomoč kakšne kvalitete je ta koda.
\item
  Formalni odzin učitelja oz. komunikacija med učiteljem in učencem.
\end{itemize}


\section{Metode in strategije pri uporabi spletnih portalov}
\label{sec:Metode_in_strategije_pri_učenju_programiranja}

Primer strategij in metod spletnega portala za učenje jave
\cite{thesisAWebP}:

\begin{itemize}
%\tightlist
\item
  Scaffolding -\textgreater{} Gradnja študentovega znanja
  pri katerem pomaga mentorja, z svojim znanjem in izkušnjami.
\item
  Bloomova taskonomija. Zakaj je pomembno vključevanje Bloomove
  taksonomije in kako jo vključujemo.
\item
  Konstruktivizem: Aktivnost študentov pri gradnji znanja. Učenje z
  eksperimentiranjem. Problemski pristop.
\end{itemize}

Kaj od katerih metod predstavlja v uporabi spletnega portala \ldots{}:

\begin{itemize}
%\tightlist
\item
  Spletni portal -\textgreater{} Scaffolding + Bloom
\item
  Naloge narejene tako, da podpirajo konstrutivno metotode
  -\textgreater{} problemski pristom
\end{itemize}

\subsection{Učne strategije}
\label{sec:učne_strategije}

\subsubsection{Model aktivnega učenja}
\label{sec:model_aktivnega_učenja}

Pri \textbf{aktivnem učenju} je to, da učenci z lastno aktivnostjo
ugotovijo, sami za sebe, kako nekaj deluje. Sami si morajo izmisliti
primere, preiskusiti lastne veščine in reševati neloge, ki so jih že
ali jih še podo spoznali. Učenje je aktivno usvajanje, je gradnja idej
in znanja. Za učenje mora biti posameznik aktivno vključen v gradnjo
svojih lastnih mentalnih modelov.

Model aktivnega učenja je sestavljen s štirih korakov.

\begin{enumerate}
\item \textbf{Sprožilec} Je predstavljena naloga kot iziv za uvod v novo
tematiko.  //Gerlič -> Motivacija.
\item \textbf{Aktivnost} Študenti izvajajo aktivnost, ki jim je bila
predstavljena v sprožilcu. Ta kora je lahko kratek ali lahko
zavzame večju del učne ure. To je odviso od vrste sprožilca in
izobraževalnih ciljev.
\item \textbf{Diskusija} sledi po koncu aktivnosti, kjer se zbere zeloten
razred, ne glede na obliko dela. V temo koraku študenti izpopolnijo
koncepte in ideje, kod del konstruktivnega učnega procesa.
\item \textbf{Povzetek} je lahko izračen v različnih oblikah, kot so
zaogrožene definicije, lahko so miselni vzorci ali povezav med
temami, ki so jih obravnavali študenti in med drugimi temami, ki se
navezujejo nanje.
\end{enumerate}

Ko se ta model izkaže za primernega, ga lahko uporabimo v številnih
učnih urah v različnih variacijah.

\subsubsection{Učenje  na daljavo}
\label{sec:Učenje_na_daljavo}


\subsection{Tipi nalog}
\label{tipi_nalog}

\subsubsection{Zapolni prazna mesta}
\label{sec:zapolni_prazna_mesta}

Tip nalog začenniku ponuja ogrodje programa, del programske kode, na
katerem dijak usvoji novo znanje in/ali lahko uporablja že pridobljeno
znanje.


\section{Kategoriziranje spletnih portalov}
\label{sec:kategoriziranje_spletnih_portalov}

\subsection{Vrsta vsebine}
\label{sec:Razvrstitev_spletnih_portalov}

Po hitrem pregledu izbranih spletnih portalov lahko ugotovimo, da je

\subsection{Programski jeziki}
\label{sec:programski_jeziki}


\section{Ovrednotenje izbranih spletnih portalov in njihove posebnosti}
\label{sec:pregled_spletnih_portalov}

\subsection{Določitev Kriterijev}
\label{sec:dolocitev_kriterijev}


\section{Možni načini uporabe spletnih portalov pri puku}
\label{sec:načini_uporabe_sp}









%%% Local Variables:
%%% mode: latex
%%% TeX-master: "diploma"
%%% End:
