%% UVOD

\chapter*{Uvod}
\addcontentsline{toc}{chapter}{\protect\numberline{}Uvod}



\chapter{Glavni naslov}

\chapter{Drugi glvnia naslov}
\label{chap:2.naslov}

neksne tekst!


\section{Prvi podnaslov}
\label{sec:1.podnaslov}


\chapter*{Uvod}
\addcontentsline{toc}{chapter}{\protect\numberline{}Uvod}


Tema na\v se diplomske naloge je pojem, s katerim se sre\v cujemo...

V strokovni literaturi...\\

Pojem diplomske naloge je uporaben na zelo \v sirokem podroju.

Delo je organizirano v tri dele. V prvem delu...
%%%%%%%%%%%%%%%%%%%%%%%%%%%%%%%%%%%%%%%%%%%%%%%%%%%%%%%%%%%%%%

% \doublespacing

%%%%%%%%%%%%%%%%%%%%%%%%%%%%%%%%%%%%%%%%%%%%%%%%%%%%%%%%%%%%%%
% Prvo poglavje
%%%%%%%%%%%%%%%%%%%%%%%%%%%%%%%%%%%%%%%%%%%%%%%%%%%%%%%%%%%%%%
% ----------------------------------------------------
% -=== Delne urejenosti in njihovi grafi pokritij ===-
% ----------------------------------------------------
\chapter{Naslov prvega poglavja}

\section[Naslov drugega razdelka]{Nekaj kratkih napotkov za delo z \LaTeX om}



\v Ce uporabljamo operacijski sistem Windows je  pred samim za\v
cetkom dela potrebno: \begin{itemize}
\item[(0)] preveriti ali imamo na ra\v cunalniku:
  \subitem in\v staliran program za branje pdf datotek, recimo Adobe
  Reader: \url{www.adobe.com} ali
  Foxit Reader: \url{http://www.foxitsoftware.com/}
  \subitem in\v staliran programa za ustvarjanje in za branje ps dokumentov
  (Ghostscript, GSview ) \url{http://www.cs.wisc.edu/~ghost/}
\item[(1)] in\v stalirati eno izmed distribucij, recimo MikTex :
  \url{http://www.miktex.org/}, \v ce je le mogo\v ce, izberimo polno
  namestitev:  \url{http://miktex.org/2.7/Setup.aspx}
\item[(2)] in\v stalirati urejevalnik besedil, recimo TeXmaker:
  \url{http://www.xm1math.net/texmaker/}
\item[(3)] poiskati dobro vzor\v cno datoteko (tako, da se ne rabimo
  obremenjevati z nastavitvami), recimo:
  \url{http://um.fnm.uni-mb.si/files/Seminar08_09/LaTeXPrimer.zip} ali
  pa uporabimo kar to datoteko, ki jo pravkar beremo.
\end{itemize}

\v Ce uporabljamo operacijski sistem Ubuntu v Synaptic package
managerju izberemo in in\v staliramo {\bf texlive} in {\bf kile} ter
poi\v s\v cemo dobro vzor\v cno datoteko tako kot v zgornjem
primeru. V ostalih operacijskih sistemih deluje vse podobno.\\

Nekaj osnovnih povezav:
\begin{itemize}
\item Glavna stran uporabnikov \TeX a:
  \url{http://tug.org/}

\item \v Clanek:
  \url{http://tug.org/notices/}

\item The Comprehensive TeX Archive Network (ime pove vse):
  \url{http://www.ctan.org/}

\item LaTeX project site:
  \url{http://www.latex-project.org/}

\end{itemize}

Distribucije:
\begin{itemize}
\item Texlive \url{http://www.tug.org/texlive/}
\item MikTex: \url{http://www.miktex.org/}
\end{itemize}

Nekaj urejevalnikov:
\begin{itemize}
\item Texworks: \url{http://www.tug.org/texworks/}
\item Kile: \url {http://kile.sourceforge.net/}
\item TeXmaker: \url{http://www.xm1math.net/texmaker/}
\item TeXnicCenter: \url{http://www.texniccenter.org/}
\item LaTeX Editor: \url{http://www.latexeditor.org/}
\item TextPad: \url{http://www.textpad.com/} (Nastavitve menija Tools
  v programu TextPad \url{
    http://www-lp.fmf.uni-lj.si/plestenjak/vaje/latex/textpad.htm})
  % \item WinEdt (licen\v cni): \url{http://www.winedt.com/}
\end{itemize}

Urejanje besedil preko interneta:
\begin{itemize}
\item MonkeyTeX: \url{http://monkeytex.bradcater.webfactional.com}
\item Texify: \url{http://www.texify.com/}
\item It's All Text! 1.3.1 (koristen dodatek za brskalnik Mozilla Firefox):\\
  \url{https://addons.mozilla.org/en-US/firefox/addon/4125}
\end{itemize}
\vspace{5mm}
Gradiva (priro\v cniki, u\v cbeniki, dobri zgledi):
\begin{itemize}
\item LaTeX2e v 128 minutah - verzija 4.20 (u\v cbenik v sloven\v s\v
  cini):
  \url{http://www-lp.fmf.uni-lj.si/plestenjak/vaje/latex/lshort.pdf}

\item priprava prosojnic in u\v cnega gradiva v \LaTeX-u, doma\v ca
  stran izred. prof. dr. Bora Plestenjaka:\\
  \url{http://www-lp.fmf.uni-lj.si/plestenjak/vaje/latex/latex.htm}
\item gradiva (dr. Andrej Taranenko):
  \url{http://matematika-racunalnistvo.fnm.uni-mb.si/dodatna_gradiva/numericna_matematika/taranenko/latex/latex01.pdf}
  in
  \url{http://matematika-racunalnistvo.fnm.uni-mb.si/dodatna_gradiva/numericna_matematika/taranenko/latex/latex02.pdf}
\item pisanje besedil v LaTeXu (dobra predloga!)  (doc. dr. Drago Bokal):
  \url{http://um.fnm.uni-mb.si/tiki-index.php?page=Pisanje+besedil+v+LaTeXu}
\item nekaj navodil za izdelavo nalog in predstavitev (doc. dr. Drago Bokal):
  \url{http://um.fnm.uni-mb.si/tiki-index.php?page=Nekaj+navodil+za+izdelavo+nalog+in+predstavitev}
\item nekaj zbranih ukazov na enem mestu
  \url{http://www.stdout.org/~winston/latex/}
\item iskanje pomo\v ci
  \subitem na novi spletni strani Oddelka za matematiko in ra\v
  cunalni\v stvo bo predvidoma ponovno za\v zivel forum, kjer bo mo\v
  zno poiskati pomo\v c
  % Forum na FNM: \url{http://latex.fnm-um.net}
  \subitem dopisni seznam za TeX (vodi Mojca Miklavec)
  \url{http://liste2.lugos.si/cgi-bin/mailman/listinfo/tex-list}
\end{itemize}

\v Se nekaj koristnih povezav:
\begin{itemize}
\item Spletna zbirka izra\v cunljivih podatkov:
  \url{http://www.wolframalpha.com/}
\item Operacijski sistem Ubuntu: \url{http://www.ubuntu.com/} in
  \url{www.ubuntu.si/}
\item Sagemath - ''veleprojekt matemati\v cne skupnosti''
  \url{http://www.sagemath.org/} in spletna razli\v cica programa
  \url{http://www.sagenb.org/}
\item Scilab (program za numeri\v cno ra\v cunanje): \url{http://www.scilab.org/}
\item Geogebra (program za dinami\v cno geometrijo): \url{www.geogebra.org/}
\item Open Office: \url{sl.openoffice.org/}
\item Inkscape (odprtokodni urejevalnik za vektorsko grafiko):\\
  \url{http://www.inkscape.org/}
\item Nekatere ''alternative'' za \LaTeX:
  \subitem Lyx \url{http://www.lyx.org/}
  \subitem Microsoft Word ali Open Office (oboje nepriporo\v cljivo
  kot alternativa)
  \subitem Scientific WorkPlace (no ja...)
\item Portal namenjen \v studentom Fakultete za naravoslovje in matematiko:\\
  \url{http://www.naravoslovna.net/}
\item Pomo\v c bri branju literature v tujem (angle\v skem) jeziku:
  \subitem Angle\v sko-slovenski matemati\v cni slovar
  \url{http://wiki.fmf.uni-lj.si/wiki/Kategorija:Angle%C5%A1ko-slovenski_matemati%C4%8Dni_slovar}
  \subitem Avtomatski prevajalnik besedil \url{translate.google.com/}
\end{itemize}

Podobno kot ostalih ve\v s\v cin se \LaTeX a najla\v zje nau\v cimo z
vajo. Pomo\v c lahko poi\v s\v cemo pri bolj izku\v senih uporabnikih,
recimo pri \v clanih oddelka za matematiko in ra\v cunalni\v stvo.



\section[Naslov drugega razdelka]{Naslov drugega razdelka}


\subsubsection{Naslov podrazdelka}

Sledi \v se primer na\v stevanja z mo\v znim sklicevanjem, npr. na Izrek \ref{iz:prviIzrek}:
\begin{enumerate}[label=(\roman{*}), ref=(\roman{*})]
\item \label{it:prva} Prva alineja je pred alinejo \ref{it:druga}.
\item \label{it:druga} Druga alineja sledi alineji \ref{it:prva}.
\end{enumerate}

\subsection{Predstavitev diplomskega dela}

Potem, ko smo kon\v cali z oblikovanjem besedila lahko na precej
preprost in hiter na\v cin pripravimo predstavitev s pomo\v cjo paketa
{\bf Beamer}\footnote{\v ze vklju\v cen v kompletno distribucijo
  MikTeX-a ali TeXlive-a}:

\url{http://latex-beamer.sourceforge.net/}\\
\url{http://en.wikipedia.org/wiki/Beamer_(LaTeX)}\\

Navodila v Sloven\v s\v cini:\\
\url{http://www-lp.fmf.uni-lj.si/plestenjak/vaje/latex/beamer_pregled.pdf}

In vzor\v cna datoteka v Slove\v s\v cini:\\
\url{http://um.fnm.uni-mb.si/files/Seminar08_09/seminar_predstavitev.zip}

\subsection{\v Se nekaj napotkov}

V kolikor pride pri odpiranju tex verzije tega dokumenta do popa\v
cenja dolo\v cenih nakov je dobro preveriti v kak\v snem formatu je
shranjena datoteka (ASCII, cp1250, utf8,...) in potem to ustrezno
nastaviti pri na\v sem urejevalniku. Za pomo\v c glede tega se lahko
obrnemo na strokovno osebje na FNM.

Ukaz za vklju\v citev paketa \verb|\usepackage[cp1250]{inputenc}| (\v
se posebej \v ce uporabljamo operacijski sistem Windows) oz. ukaz za
vklju\v citev paketa \verb|\usepackage[utf8]{inputenc}| (na primer \v
ce uporabljamo operacijski sistem Linux) nam omogo\v ca, da zapisujemo
posebne znake slovanskih abeced (\v sumnike) kot obi\v cajno (kar
preko tipkovnice, brez posebnih ukazov). Sicer moramo uporabiti
ustrezne ukaze, kot je razvidno iz tex verzije tega dokumenta.

V primeru, da prevajamo dokument direktno v format {\bf dvi} ali {\bf
  ps} lahko vstavimo grafi\v cne datoteke s kon\v
cnico \begin{Large}{\bf eps}\end{Large}, v kolikor prevajamo dokument
direktno v format {\bf pdf} lahko vstavimo grafi\v cne datoteke s
kon\v cnicami \begin{Large}{\bf jpg, png, pdf}\end{Large}.




%%% Local Variables:
%%% mode: latex
%%% TeX-master: "diploma"
%%% End: