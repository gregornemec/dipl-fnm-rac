\section{Uporaba računalnika v izobraževanju}
\label{sec:uporaba-raunalnika-v}

Model uporabe računalnika v izobraževanju je Gerlič \cite{gerlic_2000}
razdelil na tri področja. V \textbf{primarno področje} lahko uvrstimo
učenje računalništva in programiranja, saj sem prištevamo aktivnosti s
katerimi želimo uporabnike seznaniti z delovanjem in uporabo
računalnika oz. sodobno informacijsko komunikacijsko tehnologijo
(\textbf{IKT}). Računalnik je tista učna vsebina, ki jo
obravnavamo. \textbf{Sekundarno področje} predstavlja vse tiste
aktivnosti, katere so vezane neposredno na izobraževalni proces
katerega koli predmetnega področja. Računalnik in IKT nastopata kot
učno sredstvo ali pripomoček v oblikah tradicionalnih računalniško
podprtih učnih sistemov ali inteligentnih ekspertnih sistemov. V
\textbf{terciarno področje}, spadajo vse aktivnosti, ki spremljajo
izobraževanje. Sem se štejejo aktivnosti izobraževanja, vodenja in
upravljanja izobraževalnega sistema.

V tem diplomskem delu nas bo zanimalo le \textbf{primarno področje}
uporabe računalnika v izobraževanju, ki je razdeljeno v dveh pomembnih
področjih \cite{gerlic_2000}:

\begin{itemize}
\tightlist
\item kot element \textbf{splošne izobrazbe},
\item kot element \textbf{ožje strokovne} - poklicne izobrazbe
  oz. usposabljanja.
\end{itemize}

\subsection{Splošnoizobraževalno področje}
\label{sec:spološnoiz_področje}

V današnjem času se računalnik kot element splošne izobrazbe kaže kot
velika potreba oz. se zdi znanje njegove uporabe samoumevno. Že pri
najmlajših otrocih računalnik vzbuja zanimanje in interes.  Računalnik
je postal intelektualno orodje in pripomoček v vsaki sferi človekove
dejavnosti in je prodrl že dolgo nazaj tudi v šolo. Tako imenovana
\textbf{računalniška pismenost} postaja nuja in zajema vse to kar bi
človek moral znati o računalniku in to, kako je potrebno z njim
delati, da bo uspešno živel v družbi, ki je osnovana na informacijah
oz. informacijski družbi. V zvezi z definicijo in pojmovanjem
\textbf{računalniške pismenosti} se kažeta dve usmeritvi, Gerlič
\cite{gerlic_2000} navaja številne avtorje obeh usmeritev in povzema:

Prva poudarja \textbf{sposobnost računalniškega programiranja} in
opredeljuje s pojmom pismenosti sposobnost branja in pisanja
podobno kot je to značilno za jezikovno pismenost. Tako zagovorniki,
te smeri poudarjajo, da je cilj računalniške pismenosti, učenje in
veščina programiranja z novim načinom mišljenja in strategijami
ugotavljanja in popravljanja računalniških programov.

Druga smer poudarja \textbf{splošno usposobljenost} za delo z
računalnikom in to, da ni smiselno, da vsak kdo postane programer,
zaradi tega, ker se bo računalnik uporabljal v najširšem smislu v
praksi. Pomembno za učenca je da razume, delovanje računalnika in se
zaveda njegovega vpliva na razvoj družbe. Učencu moramo pomagati, da
dejanske probleme identificira in jih lahko reši že z narejeno
komercialno programsko opremo.

V zgodnjih letih sta bili značilni obe usmeritvi. Pozneje je prišlo do
preobrata leta 1987 po mednarodnem simpoziju na Univerzi v
Stanfordu. Eden od sklepov simpozija je bil ta, da se v
splošnoizobraževalne programe, več ne uči programiranja, še posebej ne
v strukturiranih verzij programskega jezika, kot je na primer
\textbf{BASIC}, temveč naj se uči uslužnostne programske opreme, kot
je urejevalnik besedil, orodja za delo z podatkovnimi bazam, grafična
orodja itd \dots . Ta dejstva je Gerlič \cite{gerlic_2000} povzel leta
2000 in je predlagal isto usmeritev z naslednjim navedkom:

\emph{``Učencem vseh stopenj želimo ob
  čim večjem številu ur praktičnega dela z računalnikom, ob določenem
  problemu in ob uporabi ustreznih komercialnih programov seznaniti z
  osnovami računalništva in informatike.''}

%Mogoče sem predrzen
Zanima nas koliko je računalniške znanosti in programiranja v
splošnoizobraževalnem področju v učnih načrtih za \textbf{OŠ} in
\textbf{SŠ}. Novi trendi in potrebe industrije kažejo na potrebo
povečanja po znanjih programiranja. Ali zaradi razširjenosti
računalniške tehnologije, ki je na vsakem koraku bi morda bilo
potrebno definicijo računalniške znanosti, ki jo je povzel Gerlič
posodobiti in ji dodati nazaj osnove znanje računalništva in
programiranja ter uporabe odprto kodne programske opreme. Predvsem pri
mlajših generacij obstaja potreba in trend, da se programiranje ponudi
v širšem kontekstu tudi na splošnem izobraževalnem nivoju.

Ne glede na to, kako bo v prihodnje definirana računalniška pismenost,
v diplomskem delu želimo pokazati, da spletni portali za
učenje programiranja zaradi tehnološkega napredka omogočajo lažjo pot
k učenju programiranja. Zato nas po pozneje zanimalo kje je umeščeno v
učnem načrtu programiranje v \textbf{OŠ} in \textbf{SŠ}.

\subsection{Strokovno izobraževalno področje}
\label{sec:strokovno_izo_podrocje}

%% Gerlič 2000 -> ima večji del tega poglavja namenjenega znanje
%% računalništva za pedagoške delavce.

Sem prištevamo vse tiste aktivnosti, s katerimi želimo udeležence
izobraževanja usposobiti na različnih ravneh tako, da se bodo ti z
računalnikom in informacijskimi sistemi ukvarjali na poklicni ravni
Lahko povemo, da sem spadajo vse ozko usmerjene računalniške in
informacijske smeri srednjih šol, visokih, višjih ter univerzitetnih
študijev. Zanima nas področje programiranja, kar je sicer predmet
strokovnega izobraževanja, vendar nas bo zanimalo programiranje na
splošnem izobraževalnem področju. Spletne portale za učenje
programiranja bomo predstavili predvsem kot vstopno točko za začetnike
in novince, katere seveda najdemo prav tako na vseh stopnjah. 

\subsection{Programiranje v OŠ}
\label{sec:Programiranje_v_OŠ}

Pouk računalništva v OŠ poteka kot \textbf{izbirni predmet} ali kot
\textbf{neobvezni izbirni predmet}. Za začetek bomo navedli kako je
definirano Računalništvo v učnem načrtu za izbirne predmete v osnovni
šoli \cite{ucni_nacrt-izbirni-os}: \emph{``Računalništvo je
  naravoslovno-tehnični izbirni predmet, pri katerem se spoznavanje in
  razumevanje osnovnih zakonitosti računalništva prepleta z metodami
  neposrednega dela z računalniki, kar odpira učencem in učenkam
  možnost, da pridobijo tista temeljna znanja računalniške pismenosti,
  ki so potrebna pri nadaljnjem izobraževanju in vsakdanjem življenju.
  ''}.

\subsubsection{Izbirni predmet računalništva}
\label{sec:izbirni_predmet_rac}

Izbirni predmeti računalništva so sestavljenih z treh predmetov in jih
učenke in učenci lahko izberejo v tretjem trilerju, v 7., 8. in/ali
9. razredu. Prvi od teh predmetov je \textbf{Računalništvo - urejanje
  besedil}, kjer si učenci pridobijo osnovna znanja, ki so potrebna za
razumevanje in temeljno uporabo računalnika. Naslednja dva predmeta
sta \textbf{Računalniška omrežja} in \textbf{Multimedija}, kjer se ta
znanja spiralno nadgradijo.

Zanima nas kje se pri teh predmetih računalništva pojavlja
programiranje. Vsak izmed teh predmetov ima operativne učne cilje tako
razdeljeno, da programiranje najdemo v tretji enoti, ki spada med
dodatne vsebine. Posamezne operativne cilje, dejavnosti in vsebino
prikazuje tabela \ref{tab:cilji_izb_prog}.

\begin{table}[!htb]
  \caption{Operativni cilji, dejavnosti in vsebine izbirnega predmeta
    računalništvo za III. dodatno enoto. \cite{ucni_nacrt-izbirni-os}}.
\label{tab:cilji_izb_prog}
\begin{tabular}{
  | p{0.33\linewidth-2\tabcolsep} |
  p{0.33\linewidth-2\tabcolsep} |
  p{0.33\linewidth-2\tabcolsep} | }
  \hline
  \rowcolor{gray!50}
  \textbf{OPERATIVNI CILJI} & \textbf{DEJAVNOSTI} & \textbf{VSEBINE} \\
  \hline

  \begin{itemize}[leftmargin=*]
    \tightlist
  \item napisati algoritem z odločitvijo, ki reši preprost vsakdanji
    problem;
  \item izdelati in spremeniti računalniški program z odločitvijo.
  \end{itemize}
  &
  \begin{itemize}[leftmargin=*]
  \item analizirati preprost problem;
  \item uporabljati osnovne korake programiranja.
  \end{itemize}
 &
  \begin{itemize}[leftmargin=*]
  \item risanje diagrama poteka za problem z odločitvijo;
  \item izdelava računalniškega programa.
  \end{itemize}
 \\
 \hline
\end{tabular}
\end{table}

Programiranje se pojavlja le kot dodatna vsebine, kar se kaže v
uresničitvi smeri računalništva, ki zagovarja \textbf{splošno
  usposobljenost}, kot smo jo opredelili v poglavju
\ref{sec:spološnoiz_področje}. Vendar se tu učiteljem računalništva
ponuja izjemna priložnost, da z učenci in učenkami naredijo korak v
osnove računalništva in programiranja pri veh treh predmetih.

\subsubsection{Neobvezno izbirni predmet računalništva}
\label{sec:neobvezno_izbirni_predmet_rac}

Opredelitev predmeta v učnem načrtu
\cite{ucni_nacrt-neobvezni-izbirni-os}, pravi, da neobvezni izbirni
predmet računalništva učence seznanja z različnimi področji
računalništva, računalniškimi koncepti ter procesi in jih ne učijo
dela z posameznimi programi. Učenci se seznanjajo tehniko in metodami
reševanja problemov, razvijajo algoritmičen način
razmišljanja. Opredelitev zadostuje prvi smeri računalniške
pismenosti, ki zagovarja \textbf{sposobnost računalniškega
  programiranja.}. Ker se v tej diplomski nalogi še posebej posvečamo
programiranju nam učni načrt tega predmeta dosti bolj ustreza in si ga
bomo podrobneje pogledali, saj nam bo pregled operativnih ciljev
služil kot vodilo pri nadaljnjem delu.  Najprej preglejmo splošne
cilje, katere povzemamo po učenem načrtu
\cite{ucni_nacrt-neobvezni-izbirni-os} in jih morajo doseči učenci:
\begin{itemize}
\tightlist
\item spoznavajo temeljne koncepte računalništva,
\item razvijajo algoritmični način razmišljanja in spoznavajo
  strategije reševanja problemov,
\item razvijajo sposobnost in odgovornost za sodelovanje v skupini ter
  si krepijo pozitivno samopodobo,
\item pridobivajo sposobnost izbiranja najustreznejše poti za rešitev
  problema,
\item  spoznavajo omejitve človeških sposobnosti in umetne
  inteligence,
\item se zavedajo omejitev računalniških tehnologij,
\item pridobivajo zmožnost razdelitve problema na manjše probleme,
\item se seznanjajo z abstrakcijo oz. poenostavljanjem,
\item spoznavajo in razvijajo zmožnost modeliranja, strokovno
  terminologijo.
\item  razvijajo ustvarjalnost, natančnost in logično razmišljanje,
\item razvijajo in bogatijo svoj jezikovni zaklad ter skrbijo za
  pravilno slovensko izražanje in strokovno terminologijo.
\end{itemize}

Neobvezni izbirni predmet je namenjen učencem 4., 5. in 6. razreda. V
učnem načrtu so operativni cilji predstavljeni tako, da so temeljni
označeni \textbf{krepko} in izbirni \emph{poševno}. Navedli bomo
tiste, ki so navezujejo na programiranje in strategije reševanja
problemov. Operativni cilji so razdeljeni na vsebinske sklope.

\textbf{Vsebina/sklop: Algoritmi}
\begin{itemize}
\tightlist
\item \textbf{razumejo pojem algoritem},
\item \textbf{znajo vsakdanji problem opisati kot zaporedje korakov},
\item \textbf{znajo z algoritmom predstaviti preprosto opravilo},
\item \textbf{algoritem predstavijo simbolno (z diagramom poteka) ali s
  pomočjo navodil v preprostem jeziku},
\item \textbf{sledijo algoritmu, ki ga pripravi nekdo drug},
\item \textbf{znajo v algoritem vključiti vejitev (če) in ponavljanje (zanke)},
\item \emph{znajo algoritem razgraditi na gradnike (podprograme)},
\item \textbf{znajo povezati več algoritmov v celoto, ki reši neki problem},
\item \emph{razumejo vlogo testiranja algoritma in vedo, da je testiranje
  orodje za iskanje napak in ne za potrjevanje pravilnosti},
\item \emph{primerjajo več algoritmov za rešitev problema in znajo poiskati
  najustreznejšega glede na dana merila},
\item \emph{znajo uporabiti nekatere ključne algoritme za sortiranje in
  iskanje},
\item \emph{poznajo osnovne algoritme za iskanje podatkov}.
\end{itemize}

\textbf{Vsebina/sklop: Programi}
\begin{itemize}
\tightlist
\item \textbf{znajo slediti izvajanju tujega programa},
\item \textbf{znajo algoritem zapisati s programom},
\item \textbf{znajo v program vključiti konstante in spremenljivke},
\item \textbf{ razumejo različne podatkovne tipe in jih znajo uporabiti v
  programu},
\item \textbf{znajo spremenljivkam spremeniti vrednost s prireditvenim
  stavkom},
\item \textbf{znajo v programu prebrati vhodne podatke in jih vključiti v
  program},
\item \textbf{znajo izpisovati vrednosti spremenljivk med izvajanjem programa
  in izpisati končni rezultat},
\item \textbf{v program vključijo logične operatorje},
\item \textbf{znajo uporabiti pogojni stavek in izvesti vejitev},
\item \textbf{razumejo pojem zanke in ga znajo uporabiti za rešitev problema},
\item \emph{razumejo kompleksnejše tipe podatkov (nizi, seznami/tabele) in
  jih znajo uporabiti v programu},
\item \textbf{prepoznajo in znajo odpraviti napake v svojem programu},
\item \emph{znajo popraviti napako v tujem programu},
\item \emph{znajo spremeniti program, da dosežejo nov način delovanja
  programa},
\item \emph{znajo rezultate naloge zapisati v datoteko},
\item \textbf{se seznanijo z dogodkovnim programiranjem},
\item \textbf{so zmožni grafične predstavitve scene (velikost objektov, ozadje, pozicioniranje)},
\item \textbf{so zmožni sinhronizacije dialogov/zvokov},
\item \textbf{so zmožni razumeti in realizirati interakcije med liki in objekt}i,
\item \textbf{ so zmožni ustvarjanja animacij}.
\end{itemize}

\textbf{Vsebina/sklop: Podatki}
\begin{itemize}
\tightlist
\item \textbf{razlikujejo podatek in informacijo},
\item \textbf{razumejo dvojiški sistem zapisovanja različnih podatkov},
\item \textbf{razumejo kodiranje podatkov},
\item \textbf{razumejo, da obstajajo podatki v različnih pojavnih oblikah
  (besedilo, zvok, slike, video)},
\item \emph{poznajo načine predstavitev določenih podatkov in odnose med
  njimi (dvojiška drevesa in grafi)},
\item \emph{vedo za stiskanje podatkov in vedo, da je stiskanje lahko brez
  izgub ali z izgubami},
\item \textbf{pojasnijo razliko med konstantami in spremenljivkami v programu},
%\item \textbf{znajo strukturirane podatke zapisati v tabele z vrsticami in
  %stolpci},
%\item \textbf{opišejo potrebo po urejanju podatkov},
\item \emph{poznajo osnovne algoritme za iskanje podatkov},
%\item \textbf{se zavedajo pomembnosti varovanja osebnih podatkov}.
\end{itemize}

\textbf{Vsebina/sklop: Reševanje problemov}
\begin{itemize}
\tightlist
\item \emph{znajo uporabiti različne strategije za reševanje problema},
\item \textbf{znajo našteti faze procesa reševanja problema},
\item \emph{znajo postavljati vprašanja in ugotoviti, kateri podatki so
  znani},
\item \emph{znajo za podano nalogo izluščiti bistvo problema},
\item \textbf{znajo najti ustrezno orodje, s katerim rešijo problem},
\item \textbf{znajo problem razdeliti na več manjših problemov},
\item \textbf{znajo načrtovati in realizirati rešitev},
\item \emph{za podano rešitev znajo oceniti posledice in vpliv na ``okolje''},
\item \emph{znajo uporabiti znano strategijo v novih okoliščinah},
\item \emph{znajo ustvariti nov algoritem za bolj kompleksne probleme},
\item \textbf{znajo učinkovito sodelovati v skupini in rešiti problem
    z uporabo informacijsko-komunikacijske tehnologije znajo ceniti
    neuspešne poskuse reševanja problema kot del poti do rešitve},
\item \emph{znajo kritično ovrednotiti rešitev in ugotoviti ali rešitev
  uspešno reši dani problem},
\item \emph{znajo kritično ovrednotiti strategijo reševanja problema},
\item \emph{zavedajo se omejitev informacijsko-komunikacijske tehnologije
  pri reševanju problemov}.
\end{itemize}

\textbf{Vsebina/sklop: Komunikacija in storitve}
\begin{itemize}
\tightlist
%\item \textbf{poznajo temeljne ideje o delovanju računalniških omrežij},
%\item \textbf{poznajo glavne storitve računalniških omrežij (e-pošta, splet
  %idr.)},
\item \textbf{znajo uporabiti ustrezna orodja in metode za iskanje po spletu},
\item \textbf{znajo uporabiti različne iskalne strategije v iskalnikih},
\item \textbf{poznajo omejitve pri rabi na spletu najdenih informacij
  (zavedajo se pojma intelektualna lastnina)},
%\item \textbf{znajo prikazati podatke na ustrezen način},
%\item \textbf{poznajo glavna varnostna priporočila v omrežjih in omrežni
  %bonton (netetika)}.
\end{itemize}

Kot smo že povedali smo nekatere cilje izvzeli, čeprav je znanje
računalniških omrežij pomembno, ga z vidika programiranja lahko
izpustimo. Lahko povemo, da nam je večina operativnih ciljev ostala,
saj smo jih lahko izključili le malo. Pridemo do spoznanja, da se v
računalniški znanosti pretežen del znanja vrti okoli programiranja in
reševanja problemov, kar je zajeto v tem učnem načrtu. Z samih ciljev
lahko spoznamo tudi samo zahtevnost snovi. Lahko si upamo napovedati,
da jih bomo s pravimi orodji kot so spletni portali za učenje
programiranj tudi lažje dosegli.

\subsection{Programiranje v SŠ}
\label{sec:Programiranje_v_SŠ}

% Dobor bi bilo, da za programiranje SŠ povzamemo splošni program
% gimnazije. OŠ ima zelo zahtevne cilje, lahko bi jih povzeli in
% dodali noto težavnosti.

%Določiti moramo, kateri srednješolski program bomo povzeli.
%Predlagam, splošno gimnazijo z 4. letnim programom rac in informatika

Pregledali bomo predmet \textbf{informatike}, splošnega Gimnazijskega
programa in \textbf{Računalništvo} Tehniške Gimnazije.

\subsubsection{Informatika - Splošni gimnazijski program}
\label{sec:informatika_splošni_gim_program}

%Koliko je ur informatike?
Predmet \textbf{informatike} se poučuje v prvem letniku splošne,
klasične in strokovne Gimnazije. če povzamemo opredelitev z učnega
načrta \cite{ucni_nacrt-informatika-gim}, ta pravi
naslednje. \emph{``Informatika je splošnoizobraževalni predmet, pri
  katerem se teorija poznavanja in razumevanja osnovnih zakonitosti
  informatike prepleta z metodami neposrednega iskanja, zbiranja,
  hranjenja, vrednotenja, obdelave in uporabe podatkov z digitalno
  tehnologijo z namenom oblikovanja rele- vantnih informacij za
  dograjevanje lastnega znanja in za njegovo predstavitev oziroma
  posredovanje drugim.''}

Poučevanje Informatike obsega \emph{70ur} v 1.letniku, izbirni predmet
obsega \emph{210ur} in maturitetni predmet ima \emph{70 + 210ur}, pri
čemer je \emph{70ur} namenjenih projektnemu delu.

Cilji vsebine predmeta so razporejeni na dve ravni:

\begin{itemize}
\item \textbf{splošna znanja}, v katerih dijaki razvijajo temeljnje
  digitalne kompetence, ki so potrebne za učinkovito uporabo digitalne
  tehnologije pri razvijanju lastnega znanja in za njegovo
  predstavitev oziroma posredovanje drugim;
\item \textbf{posebna znanja}, s katerimi dijaki znanje, veščine,
  osebnostne in vedenjske značilnosti,prepričanja, motive in druge
  zmožnosti splošnega znanja spiralno nadgradijo.
\end{itemize}

Za nas pomembne enote najdemo na drugi ravni, \textbf{posebnih znanj}
v tematskem sklopu \textbf{obdelava podatkov}. Zbrali bomo cilje,
kateri izpostavljajo programiranje. Dijaki:

\begin{itemize}
\tightlist
\item \textbf{Računalniška obdelava podatkov}
  \begin{itemize}
    \tightlist
  \item poznajo vlogo računalniškega programa in razložijo pomen programiranja;
  \end{itemize}
\item \textbf{Algoritmi}
  \begin{itemize}
    \tightlist
  \item  opredelijo algoritem in poznajo temeljne zahteve za algoritem,
  \item poznajo temeljne gradnike algoritma, razvijejo algoritem za
    problem z vejiščem in zanko (do 15 gradnikov),
  \item uporabijo diagram poteka in uporabljeno rešitev utemeljijo,
  \item analizirajo algoritem, ki reši zahtevnejši problem, in ga
    ovrednotijo;
  \end{itemize}
\item \textbf{Programski jezik}
  \begin{itemize}
    \tightlist
  \item opredelijo programski jezik in razložijo njegovo funkcijo,
  \item poznajo temeljne gradnike izbranega programskega jezika,
  \item razložijo njihovo funkcijo in razlago ponazorijo s primeri,
  \item opredelijo strukturirano, objektno in dogodkovno
    programiranje,
    \item ločijo med prevajalnikom in tolmačem in razliko razložijo;
  \end{itemize}
\item \textbf{Programiranje}
  \begin{itemize}
    \tightlist
  \item za dani algoritem izdelajo računalniški program,
  \item opredelijo dokumentiranje programa in razložijo njegov pomen,
  \item analizirajo program in ovrednotijo rezultate, dobljene s
     programsko rešitvijo;
  \end{itemize}
\end{itemize}

%Ali naj še dodam tehniško gimnazijo.
\subsubsection{Računalništvo - Tehniška gimnazija}

Cilje učenja programiranja pri tem maturitetnem predmetu najdemo pod
poglavjem \textbf{Programski jeziki in programiranje}. Cilji so
naslednji, dijaki \cite{ucni_nacrt-teh-gim}:

\begin{itemize}
\item poznajo pomen načrtovanja in sistematične gradnje programa,
\item poznajo pojem algoritma in opredelijo njegove lastnosti
  (razumljivost, končnost, enoumnost, razčlenjenost),
\item opišejo načine zapisa algoritma in jih prikažejo na primeru,
  izdelajo algoritem za lažji problem,
\item opredelijo pojem programskega jezika in razložijo njegovo
  funkcijo,
\item poznajo različne vrste programskih jezikov in jih
  razvrstijo po namenu in uporabi,
\item poznajo temeljne gradnike postopkovnega programskega jezika, razložijo njihove funkcije in
razlago ponazorijo s primeri (zaporedje, vejitev, iteracija),
\item opredelijo strukturirano in objektno programiranje,
\item ločijo med prevajanjem in tolmačenjem in razliko razložijo,
\item poznajo osnovno razvojno okolje,
\item poznajo pojme deklaracija, inicializacija, postopek, konstanta,
  spremenljivka, rezervirana beseda, operator, prioriteta,...
\item ločijo med pojmi izvorna koda, izvršljiva koda in vmesna (byte)
  koda,
\item pridobljeno znanje o tipih in algoritmih prenesejo v programski
  jezik in samostojno rešujejo probleme s pomočjo višjega programskega
  jezika Java,
\item poznajo in uporabijo osnovne in sestavljene tipe podatkov,
\item uporabijo pogojna stavka (if, switch),
\item iteracijo realizirajo z zankami (do, for, while),
\item uporabijo tokove podatkov,
\item pripravijo testne podatke, testirajo delovanje programa in beležijo rezultate testiranj,
\item uporabijo razhroščevalnik,
\item napišejo sled programa,
\item izdelajo dokumentacijo programa,
\item uporabijo metode za delo z objekti razredov Math, String, StringBuffer, Integer, Double ...
\item napišejo definicijo razreda(lastnosti, metode ...),
\item deklarirajo in uporabijo objekte,
\item poznajo vlogo in način izvajanja konstruktorja,
\item uporabijo enkapsulacijo, dedovanje in polimorfizem,
\item poznajo načine za prestrezanje in obravnavo izjem,
\item napišejo programe za enostavne probleme iz okolja,
\item analizirajo program in ovrednotijo rezultate, dobljene s programsko rešitvijo (različni
\item algoritmi urejanja podatkov, različni načini zapisovanja podatkov v datoteke ...),
\item predstavijo delovanje programa,
\item analizirajo in kritično vrednotijo rešitve,
\item \emph{poznajo zahtevnejše tehnike programiranja,}
\item \emph{napišejo program z uporabo zahtevnejših tehnik programiranja.}
\end{itemize}

Cilji določajo zelo podrobno obravnavo snovi programskih jezikov in
programiranja, ki na prvi pogled presega uporabo spletnih portalov za
učenja programiranja.

%Ali so spletni portali koristni pri uporabi z strokovnimi predmeti.

%%% Local Variables:
%%% mode: latex
%%% TeX-master: "../diploma"
%%% End:
