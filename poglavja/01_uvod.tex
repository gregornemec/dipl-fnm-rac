\section{Uvod}
\label{sec:Uvod}

% Originalna dispozicija: S trendom popularizacije programiranja, so
% se v zadnjem času pojavili spletni portali, kot je Codeacademy, ki
% omogočajo spletno učenje programiranja in kodiranja. V diplomskega
% dela določite kriterije in z njimi ovrednotite posamezne portale.
% Raziščite možne načine uporabe spletnih portalov in podajte ideje
% kako bi jih lahko umestili v pouk računalništva v OŠ in SŠ.

V svetovnem merilu se pojavlja trend po popularizaciji programiranja
in kodiranja. V zadnjem obdobju so se na spletu pojavili številni
portali, kot je na primer
\emph{\href{https://www.codecademy.com/}{Codeacademy}}
\cite{web:codeacademy}, ki ponujajo učenje programiranja. Zanima nas
kateri so ti portali, katere vsebine ponujajo, na kakšen način so
vsebine predstavljene in ali so uporabni za uporabo pri pouku ter na
kakšen način bi jih lahko uporabili.

Najprej bomo raziskali kako je definirana računalniška pismenost in
uporaba računalnika pri pouku. Ogledali si bomo kje se uči
programiranja na osnovni (\textbf{OŠ})in srednji šoli
(\textbf{SŠ}). Pri katerih izbirnih vsebinah, predmetih in kakšna je
vsebina, ki jo predvideva učni načrt. Podrobneje bomo preučili vse
tiste pojme, ki se pojavljajo v računalniški znanosti in
programiranju, ki nam bodo pomagali bolje razumeti spletne portale in
vsebino, ki jo predstavljajo.  Preučili bodo tudi sodobne pristope in
strategije, ki se uporabljajo pri učenju programiranja, saj bomo tako
lažje ocenili ali jih spletni portali tudi znajo upoštevati.

Zanimali nas bodo \textbf{novinci} in njihove težave pri začetnih
korakih učenja programiranja. Ko govorimo o novincih imamo v mislih
vse tiste učence, dijake in študente, ki se šele srečujejo s
programiranjem, ne glede na spol. Prav tako nas bodo zanimali
\textbf{mentorji} s katerim imamo v mislih učitelje in profesorje.

V ta namen bomo pregledali nastanek spletnih portalov za namen učenja
programiranja, ki so se pojavila v akademskem okolju na posameznih
univerzah. Zanimal nas bo razlog za nastanek takšnih spletni portalov
na univerzah, zato bomo pregledali literaturo in poskusimo ugotoviti,
zakaj in kako se na višje šolskem področju uporabljajo spletne
tehnologije za poučevanje programiranja in kako so skušale premostiti
nekatere težave, ki jih imajo novinci pri učenju programiranja.

Izluščili bomo predlagane rešitve za uporabo spletnih tehnologij pri
učenje programiranja. Na podlagi pregledanega bomo lahko določili
kriterije s katerimi bomo lahko klasificirali spletne portale in jih
tudi ovrednotili ter uspešno umestili v pouk.

%Vpelji novinec, učenec, dijak isto za mentorja. 

%%% Local Variables:
%%% mode: latex
%%% TeX-master: "../diploma"
%%% End:

