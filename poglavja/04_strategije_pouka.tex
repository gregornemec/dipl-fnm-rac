\section{Pristopi in strategije poučevanja}
\label{sec:aktivno_resevanje_prob}

V naslednjem poglavju raziščemo, kaj so sodobni pristopi in značilne
strategije pri poučevanju računalniške znanosti in
programiranja. Izpostavili bomo \textbf{model aktivnega učenja} in
\textbf{strategijo reševanja problemov}. Z razumevanjem tega bomo v
nadaljevanju lažje ocenili, ali uporaba spletnih portalov vzpodbuja oba
pristopa.

\subsection{Model aktivnega učenja}
\label{sec:model-aktivn-uenja}

Vsak pouk računalništva naj bi imeti modelno zgradbo in bi upošteval
naslednji načeli:
\begin{itemize}
\tightlist
\item naj vzpodbuja študente s pozitivno naravnanim poukom in jim naj
  omogoča okolje, kjer najdejo pomoč.
\item Pouk računalništva naj bo grajen na konstruktivnih metodah
  poučevanja in aktivnem učenju.
\end{itemize}

\textbf{Konstruktivizem} je kognitivna teorija, ki preučuje naravo
procesov učenja. Po tem principu naj bi učenci konstruirali novo
znanje na osnovi preurejanja in izpopolnjevanja že obstoječega
znanja. Znanje se gradi na obstoječih mentalnih strukturah in na
odzivu, ki ga dobi učenec iz učnega okolja. Mentalne strukture so
grajene korak za korakom, ena za drugo, lahko pa s to metodo pride
tudi do sestopanja ali slepih koncev. Proces je povezan s Piagetovim
mehanizmom asimilacije \cite{guideTCS}.

Pri \textbf{aktivnem učenju} je najpomembnejše to, da učenci z lastno
aktivnostjo ugotovijo, kako nekaj deluje. Sami si morajo
izmisliti primere, preiskusiti lastne veščine in reševati neloge, ki
so jih že ali jih še bodo spoznali. Učenje je aktivno usvajanje, je
gradnja idej in znanja. Za učenje mora biti posameznik aktivno
vključen v gradnjo svojih lastnih mentalnih modelov. Model aktivnega
učenja je sestavljen iz štirih korakov \cite{guideTCS}.

\begin{itemize}
\tightlist
\item \textbf{Sprožilec} je naloga, ki predstavlja iziv za uvod v novo
tematiko.
\item \textbf{Aktivnost.} Študenti izvajajo aktivnost, ki jim je bila
predstavljena v sprožilcu. Ta korak je lahko kratek ali lahko
zavzame večji del učne ure. To je odviso od vrste sprožilca in
izobraževalnih ciljev.
\item \textbf{Diskusija} sledi po koncu aktivnosti, kjer se zbere celoten
razred, ne glede na obliko dela. V tem koraku študenti izpopolnijo
koncepte in ideje kot del konstruktivnega učnega procesa.
\item \textbf{Povzetek} je lahko izražen v različnih oblikah, kot so
zaokrožene definicije, lahko so miselni vzorci ali povezave med
temami, ki so jih obravnavali študenti, in med drugimi temami, ki se
navezujejo nanje.
\end{itemize}

Ko se ta model izkaže za primernega, ga lahko uporabimo v številnih
učnih urah v različnih variacijah. Zanima nas, ali znajo pristop
aktivnega učenja spletni portali upoštevati.

\subsection{Strategije reševanja problemov}
\label{sec:strategije_reševanja_problemov}
%Preveriti moram knjigo Norbert Jaoušovec
Programiranje je preces, pri katerem rešujemo probleme, zato je
reševanje problemov v središču poučevanja računalniške
znanosti. Reševanje problemov je zahteven mentalni proces. Če na
spletu pobrskamo za strategije reševanja problemov, lahko hitro
ugotovimo, da obstajajo različne strategije. Med njimi lahko najdemo
 \textbf{abstrakcijo, analogijo, brainstorming, deli in
  vladaj} in mnoge druge \cite{book:jausovec}. Proces in tehnike
reševanje problemov se uporablja v mnogih tehničnih in znanstvenih
disciplinah \cite{guideTCS}.

V nekaterih primerih učenci sami razvijejo strategijo, s katero rešijo
nek problem. Otroci si na primer sami izmislijo preprosto seštevanje in
odštevanje, še preden se to učijo pri pouku
matematike. Toda brez formalne podpore za učinkovito strategijo
reševanja problemov spodleti še tako inovativnemu učencu tudi pri
preprostih strategijah, kot je \textbf{preizkus in napaka}. Zato je
pomembno, da se uči strategij za reševanje problemov.

%\subsection{Proces reševanja problemov}
%\label{sec:proces_reševanja_problemov}

Vsak osnovni proces, ki se ukvarja z reševanjem problemov, ne glede na
znanstveno disciplino, se začne z opisom problema. Vsak problem se
navadno zaključi z neko rešitvijo, ki je v nekaterih primerih izražena
z \textbf{zaporedjem korakov} ali \textbf{algoritmom}. V računalnik
algoritem zapišemo s kodo nekega programskega jezika. Zapisan
algoritem testiramo tako, da kodo prevedemo v strojni jezik in jo
izvedemo. Za pravilno delovanje programa primerjamo vhodne in želene
izhodne podatke. Preden pridemo od opisa problema do podane rešitve,
moramo prehoditi kar nekaj težkih korakov. Na te vmesne korake lahko
gledamo kot na procese odkrivanja, zato lahko na reševanje problemov
gledamo tudi kot na kreativen, umetniški proces. Splošno priznani
koraki reševanja procesov so naslednji \cite{guideTCS}:

\begin{enumerate}
\tightlist
\item \emph{Analiza problema}. Najprej je pomembno, da razumemo, kaj je
  problem in ga znamo identificirati. Če tega ne znamo, ne moremo
  priti do nobene rešitve.
\item \emph{Alternativne rešitve}. Razmišljamo o alternativnih
  rešitvah, kako bi lahko rešili nek problem.
\item \emph{Izbira pristopa}. Izberemo primeren pristop, kako rešiti problem.
\item \emph{Razgradnja problema}. Problem razgradimo na manjše podprobleme.
\item \emph{Razvoj algoritma}. Algoritem razvijamo po korakih, ki smo
  jih določili v podproblemih.
\item \emph{Pravilnost algoritma}. Preverjanje pravilnosti algoritma.
\item \emph{Učinkovitost algoritma}. Izračunamo učinkovitost algoritma.
\item \emph{Refleksija}. Naredimo refleksijo in analizo za pot, ki smo
  jo naredili pri reševanju problema in naredimo zaključek s tem, kar
  lahko izboljšamo za naslednji problem, ki ga bomo reševali.
\end{enumerate}

Točen recept, kako se lotiti reševanja, ne obstaja. Učencem lahko le
pokažemo nekatere metode in strategije, ki jim lahko pomagajo pri
reševanju problemov. Da bi bolje znali oceniti, kako izrazito spletni
portali upoštevajo korake strategije reševanja problemov, poglejmo še podrobneje nekatere pomembne korake in kako se z njimi spopadajo
novinci.

%Od tu naprej je mogoče prepodrobno in ni potrebno vsega spodnjega.

\subsubsection{Razumevaje problema}
\label{sec:razumevanje problema}

Razumevanje problemov je prva stopnja v procesu reševanja
problemov. Pri reševanju algoritemskih nalog moramo najprej
prepoznati, kaj so vhodni podatki in kateri podatki naj bi bili
izhodni. Če znamo povedati, kaj bodo vhodni podatki, razumemo tudi
bistvo samega problema.

\subsubsection{Načrtovanje rešitve}
\label{sec:načrtovanje_rešitve}

Novinci se spopadajo z največjimi težavami na začetni stopnji
načrtovanja rešitve za nek problem. V nadaljevanju so predstavljene
tri strategije, ki jih lahko uporabimo pri tem koraku reševanja
problema.

\begin{description}
\item [Definicija spremenljivk problema:] Pri rešitvi problema
  si pomagamo tako, da ugotovimo, kaj morajo biti vhodni in kateri bojo
  izhodni podatki. S tem razjasnimo problem. V naslednjem koraku
  definiramo \textbf{spremenljivke}, ki so potrebne za rešitev
  problema.
\item [Postopno izboljševanje (\emph{ang. Stepwise
      Refinement)}:]Po tej metodi nas najprej zanima celoten pregled
  strukture problema in odnosi med posameznimi deli. Zatem se šele
  poglobimo v specifično in kompleksno implementacijo posameznih podproblemov. Postopno izboljševanje je metodologija, ki poteka od
  \textbf{zgoraj navzdol}, torej od splošnega k specifičnemu. Drugačen
  pristop je od \textbf{spodaj navzgor}. Za oba pristopa velja, da drug
  drugega dopolnjujeta. V obeh primerih je problem razdeljen na manjše
  podprobleme ali naloge. Glavna razlika med obema je mentalni
  proces, ki je potreben za en ali drugi pristop. V nadaljevanju se
  posvetimo samo pristopu od \textbf{zgoraj navzdol}. Rešitev, ki jo
  poda \textbf{postopno izboljševanje}, ima modularno obliko, ki:
  \begin{enumerate}
    \tightlist
  \item jo lažje razvijamo in preverjamo,
  \item jo lažje beremo in
  \item nam omogoča, da uporabljamo posamezne podrešitve tudi za
    reševanje drugih problemov.
  \end{enumerate}
\item [Algoritemski vzorci:] Združujejo
  matematični pogled in elemente načrtovanja. Vzorec podaja načrt na
  rešitev, s katero se srečamo mnogokrat. Algoritemski vzorci so
  primeri elegantnih in učinkovitih rešitev problemov in predstavljajo
  abstraktni model algoritemskega procesa, ki ga lahko prilagodimo
  in ga integriramo v rešitve drugim problemom.

  Pri tem procesu lahko nastopi težava prepoznave vzorca algoritma pri
  novincih, saj ti niso sposobni prepoznati podobnosti med posameznimi
  algoritmi ali ne znajo prepoznati bistva problema, njihove posamezne
  komponente in razmerja med njimi, da bi lahko rešili nove
  probleme. V takih primerih novinci radi ponovno izumijo že njim
  poznane rešitve, ki bi jih lahko uporabili. Te težave navadno
  nastanejo zaradi slabe organizacije sistematike znanja o algoritmih.

  Proces reševanja problemov z algoritemskim vzorcem se navadno začne
  s prepoznavanjem komponent, ki vodijo k rešitvi, in iskanjem podobnih
  problemov, na katere že imamo znane rešitve. Zatem 
  vzorec prilagodimo rešitvi problema in ga vstavimo v celotno
  rešitev. V večini primerov je treba vstaviti več različnih
  vzorcev, da dobimo neko novo rešitev.
\end{description}

\subsubsection{Preverjanje rešitve}
\label{sec:preverjanje_rešitve}
Ko imamo pripravljeno rešitev, moramo preveriti, ali je ta
pravilna. Pogled na preverjanje pravilnosti rešitve je lahko
teoretične in praktične narave. Razhroščevanje (\emph{ang. debugging})
spada me vrsto aktivnosti, ki nam pomaga pri ugotavljanju pravilnosti
rešitve. Splošno velja, da proces razhroščevanja s programom, ki nam
pomaga razhroščevati (ang. debugger), ali brez njega poglablja
razumevanje računalniške znanosti. S tem ko učenci razmišljajo, kako
bodo preverjali, ali njihov program deluje pravilno, hkrati v njih
poteka miselni proces refleksije o tem, kako so implementirali določen
program in kako ga bojo morebiti morali spremeniti.

Na nivoju do srednje šole uporabljamo praktične metode ugotavljanja
pravilnosti programa, kot je razgroščevanje. Ko želimo znanje
pravilnosti delovanja poglobiti, se lahko lotimo tudi teoretične
analize.

\subsubsection{Refleksija}
\label{sec:refleksije}

Refleksija je mentalni proces ali obnašanje, ki nam omogoča, da neko
delovanje analiziramo in o njem tudi premislimo. Refleksija je
pomembno orodje v splošnem učnem procesu, prav tako spada med
kognitivne procese višjega reda. Z refleksijo učenec dobi priložnost,
da stopi korak nižje in premisli o svojem razmišljanju in tako
izboljša veščino reševanja problemov. Refleksivno razmišljanje je
proces, ki zahteva veliko časa in vaje. Med procesom reševanja
problemov lahko refleksijo uporabimo na različnih stopnjah.

\begin{itemize}
\tightlist
\item \emph{Pred} reševanjem problemov. Ko problem preberemo in že
  načrtujemo rešitev, se splača uporabiti refleksijo in razmisliti o
  tem, ali smo morda že reševali podoben problem in temu primeren
  vzorec algoritma.
\item \emph{Med} reševanjem problema. Ko rešujemo problem, refleksija
  služi kot pregled, kontrola in nadzor. Na primer, ko nastopijo
  težave pri načrtovanju rešitve ali morda zaznamo težavo ali
  napako. Temo procesu lahko pravimo \textbf{refleksija v akciji}.
\item \emph{Po} reševanju problema. Ko že najdemo rešitev, ki deluje,
  nam refleksija služi kot orodje, s katerim pregledamo učinkovitost
  delovanja. Pregledamo strateške odločitve, ki so bile sprejete med
   načrtovanjem rešitve.
\end{itemize}

Refleksija je kreativni proces in je pomemben tako za učenca kot za
učitelja.


%%% Local Variables:
%%% mode: latex
%%% TeX-master: "../diploma"
%%% End:
