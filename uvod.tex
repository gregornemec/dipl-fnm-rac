\section{UVOD}
\label{sec:Uvod}

% Originalna dispozicija: S trendom popularizacije programiranja, so
% se v zadnjem času pojavili spletni portali, kot je Codeacademy, ki
% omogočajo spletno učenje programiranja in kodiranja. V diplomskega
% dela določite kriterije in z njimi ovrednotite posamezne portale.
% Raziščite možne načine uporabe spletnih portalov in podajte ideje
% kako bi jih lahko umestili v pouk računalništva v OŠ in SŠ.

V svetovnem merilu se pojavlja trend po popularizaciji programiranja
ali kodiranja. V zadnjem obdobju so se na spletu pojavili številni
portali, kot je na primer \emph{CodeAcademy}, ki ponujajo učenje
programiranja. 

%%ToDO:
%%Kdaj točno se srečajo dijaki s programiranjem in kateri
%%srednješolski program bomo pregledali podrobneje?

%\TODO[inline]{Testiranje todojev} %Uporaba za dodajanje opomb v vrstici!
%\TODO{Testiranje novega ukaza!} %Uporaba za testiranje opomb na robu!

V večini se učenci prvič srečajo s pojmi programiranja pri izbirnih
predmetih \emph{Urejanje besedil, Multimedija in Računalniška
  omrežja}. Dijaki se srečajo s programiranjem v 1. letniku pri
predmetu informatike. Posebnost so strokovni programi, katerim je
osnova računalništvo. Zanimali nas bodo novinci in njihove težave pri
začetnih korakih učenju programiranja. Torej vsi učenci dijaki in
študentu, ki se šele srečujejo s programiranjem.

Najprej se je uporaba spletne tehnologija in nastanek spletnih
portalov za namen učenja programiranja pojavila v akademskem okolju na
posameznih univerzah. Zanimal nas razlog za nastanek takšnih okolij
na univerzah, zato smo pregledali literaturo in poskušali ugotoviti,
zakaj in kako se na višje šolskem področju uporabljajo spletne
tehnologije za poučevanje programiranja.

Izluščili smo predlagane rešitve za uporabo spletnih tehnologij pri
učenje programiranja. Spoznali smo kaj so osnovni koncepti
programiranja s katerimi se srečajo novinci in katere so strategije in
metode, ki se pri učenju programiranja uporabljajo.

Na podlagi pregledanega smo lahko določili kriterije in tako smo
klasificirali ter ovrednotili spletne portale. Najbolj bojo zanimivi
tisti spletni portali, ki ponujajo številne programske jezike, dober
urejevalnik besedil, zaganjanje napisane programske kode in neko
obliko odziva, ki uporabniku omogoča odkrivanje napak.

%%Kateri so programski jeziki primerni za učenje programiranja in
%%kateri so za to določeni v RS.

Ogledali smo is kje se uči programiranja na osnovni (\textbf{OŠ})in
srednji šoli (\textbf{SŠ}). Pri katerih izbirnih vsebinah, predmetih
in kakšna je vsebina, ki jo predvideva učni nart. Uporabo spletnih
portalov smo skušali umestiti v pouk OŠ in SŠ tako, da bo njihova
uporaba najbolj koristna in smiselna.


%%% Local Variables:
%%% mode: latex
%%% TeX-master: "diploma"
%%% End:
