\begin{thebibliography}{9}

% NOTE: Citiranje sem povzel po ieee z dokumenta ieeecitationref.pdf.
% NOTE: Spodaj dodajam
%
%nekaj uporabljenih primerov člankov, ki jih
% NOTE: bom potrebovalmm

% TODO: Prosi za pomoč pri oblikovanju in uporabi literature.

\bibitem{model_uporabe_rac_izo-web}
  Martina Fefer,
  \emph{Uporaba informacijske-komunikacijske  tehnologije v osnovnih
    šolah s prilagojenim programom},
  Univerza v Mariboru - Fakulteta za naravoslovje in matematiko, Maribor,
  1999. Pridobljeno 4.4. 2016, iz \url{http://student.pfmb.uni-mb.si/~dgunze/diplomske/d2/s6.html}.

\bibitem{gerlic_2000}
  Gerlič, Ivan, \emph{Sodobna informacijska tehnologija v
    izobraževanju}, DZS, Ljubljana, 2000.

\bibitem{klemencic_2011}
  Klemenčič M., \emph{Uporaba računalniškega programa ``Postani
    matematični mojster'' pri pouku matematike.}, Pedagoška fakulteta,
  Ljubljana, 2011.

\bibitem{LTProg01}
  Anthony Robins, Janet Rountree, and Nathan Rountree,
  ''Learning and Teaching Programming: A Review and
  Discussion'' v \emph{Comuper Science education}, vol 13, No. 2,
  2003, pp. 137 - 172.
  % Univerza: Computer Science, University of Otago, Dunedin, New
  % Zealand.

\bibitem{ITaLCP_DistanceEdu}
  S.C. Ng, S.O Choy, R. Kwan, S.F. Chan,
  ''A Web-Based Environment to Improve Teaching and Learning of
  Computer Programming in Distance Education'', \emph{ICWL'05
    Proceedings of the 4th international conference on Advances in
    Web-Based Learning}, 2005

\bibitem{mentalModels}
  L. Ma, J. D. Ferguson, M. Roper, I.Ross, M. Wood,
  ''A web-based learning model for improving programming students' mental
  models'', v \emph{Proceedings of the 9th annual conference of the subject
  centre for information and computer sciences}, HE Academy, 2008
  pp. 88-94.

\bibitem{guideTCS}
  O. Hazzan, T. Lapidot, N. Ragonis,
  \emph{Guide to Teaching Computer Science}, Springer, 2011.

\bibitem{thesisAWebP}
  Nghi Truong,
  \emph{A web-based programming environment for novice programmers},
  Queensland University of Technology, Australia, 2007.

\bibitem{ucni_nacrt-izbirni-os} Vladimir Batagelj et al., \emph{UČNI
    načrt, Izbirni predmet: Program osnovnošolskega izobraževanja,
    Računalništvo}, Ministrstvo za šolstvo, znanost in šport: Zavod RS
  za šolstvo, Ljubljana, 2002. Pridobljeno 2.4.2016 iz,
  \url{http://www.mizs.gov.si/fileadmin/mizs.gov.si/pageuploads/podrocje/os/devetletka/predmeti_izbirni/Racunalnistvo_izbirni.pdf}


\bibitem{ucni_nacrt-neobvezni-izbirni-os} Radovan Kranjc et al., \emph{UČNI
    načrt, Program osnovnošolskega izobraževanja, Računalništvo:
    neobvezni izbirni predmet}, Ministrstvo za šolstvo, znanost in
  šport: Zavod RS za šolstvo, Ljubljana, 2002. Pridobljeno 2.4.2016
  iz,
  \url{http://www.mizs.gov.si/fileadmin/mizs.gov.si/pageuploads/podrocje/os/devetletka/program_razsirjeni/Racunalnistvo_izbirni_neobvezni.pdf}.

\bibitem{ucni_nacrt-informatika-gim} Wechtersbach Rado, \emph{UČNI
    načrt, Informatika [Elektronski vir]: gimnazija : splošna,
    klasična, strokovna gimnazija : obvezni predmet (70 ur), izbirni
    predmet (210 ur), matura (70 + 210 ur)t}, Ministrstvo za šolstvo,
  znanost in šport: Zavod RS za šolstvo, Ljubljana, 2008. Pridobljeno
  2.4.2016 iz,
  \url{http://www.mss.gov.si/fileadmin/mss.gov.si/pageuploads/podrocje/ss/ programi/2008/Gimnazije/UN__INFORMATIKA_gimn.pdf}.

\bibitem{ucni_nacrt-teh-gim} Predmetna komisija Tea Lončarič et al.,
  \emph{Računalništvo [Elektronski vir] : gimnazija, tehniška
    gimnazija : izbirni strokovni maturitetni predmet (280 ur)},
  Ministrstvo za šolstvo, znanost in šport: Zavod RS za šolstvo,
  Ljubljana, 2010. Pridobljeno 2.4.2016 iz,
  \url{http://eportal.mss.edus.si/msswww/programi2010/programi/media/pdf/un_gimnazija/tehniska-gimnazija/UN_Racunalnistvo.pdf}.

\bibitem{wiki:computer_program} Wikipedia contributors, \emph{Computer
    program}, Wikipedia, The Free Encyclopedia. Pridobljeno 25.4.2016
  iz,
  \url{https://en.wikipedia.org/wiki/Computer_programming}.


\bibitem{wiki:algorithem} Wikipedia contributors, \emph{Algorithem},
  Wikipedia, The Free Encyclopedia. Pridobljeno 25.4.2016 iz,
  \url{https://en.wikipedia.org/wiki/Algorithm}.

\bibitem{web:coder} Jonah Bitautas, \emph{The Differences Between
    Programmers and Coders}, Workfunc. Pridobljeno 26.4.2016 iz,
  \url{http://workfunc.com/differences-between-programmers-and-coders/}.

\bibitem{shaums} Carl Reynolds, Paul Tymann, \emph{Principles of Computer science},
  McGraw-Hill, London, 2008.

\bibitem{OO-JS} Stoyan stefanov, Kumar Chetan Sharman, \emph{Object-Oriented Java Script,
    Second edition}, Packt Publishing, Ltd, Birmingham, 2013.

\bibitem{wiki:OO} Wikipedia contributors, \emph{Object-oriented
    programming}, Wikipedia, The Free Encyclopedia. Pridobljeno
  26.4.2016 iz,
  \url{https://en.wikipedia.org/wiki/Object-oriented_programming}.

\bibitem{wiki:java} Wikipedia contributors, \emph{Java(programming
    language)}, Wikipedia, The Free Encyclopedia. Pridobljeno
  27.4.2016 iz,
  \url{https://simple.wikipedia.org/wiki/Java_%28programming_language%29}.

\bibitem{wiki:cpp} Wikipedia contributors, \emph{C++}, Wikipedia, The
  Free Encyclopedia. Pridobljeno 27.4.2016 iz,
  \url{https://en.wikipedia.org/wiki/C%2B%2B}.

\bibitem{wiki:python} Wikipedia contributors, \emph{Python(programming
    langugage)}, Wikipedia, The
  Free Encyclopedia. Pridobljeno 30.4.2016 iz,
  \url{https://en.wikipedia.org/wiki/Python_%28programming_language%29}.

\bibitem{web:PTHardWay}
  Zed A. Shaw, \emph{Learn Python the Hard Way}. Pridobljeno 2.5.2016
  iz, \url{http://learnpythonthehardway.org/book/}.

\bibitem{wiki:tutorials} Wikipedia contributors, \emph{Tutorial},
  Wikipedia, The Free Encyclopedia. Pridobljeno 5.6.2016 iz,
  \url{https://en.wikipedia.org/wiki/Tutorial}.

\bibitem{web:TPythonTut} The Python Software Foundation, \emph{The
    Python tutorial}. Pridobljeno 6.6.2016 iz,
  \url{https://docs.python.org/3/tutorial/index.html}.

\bibitem{web:multimediaL} Richard E. Mayer, \emph{Principles for multimedia learning}. Pridobljeno 6.6.2016 iz,
  \url{http://hilt.harvard.edu/blog/principles-multimedia-learning-richard-e-mayer}.

\bibitem{web:udemy} \emph{Udemy}. Pridobljeno 6.6.2016 iz,
  \url{https://www.udemy.com}.

\bibitem{web:pythonfiddle} Python fiddle, \emph{Python Cloud
    IDE}. Pridobljeno 6.6.2016 iz,
  \url{http://pythonfiddle.com/}.

\bibitem{web:cloud9} \emph{Cloud9}. Pridobljeno 6.6.2016 iz,
  \url{https://c9.io/}.

\bibitem{web:codeenvy} \emph{Codenvy}. Pridobljeno 6.6.2016 iz,
  \url{https://codenvy.com/}.

\bibitem{web:w3school} \emph{w3school}. Pridobljeno 6.6.2016 iz,
  \url{http://www.w3schools.com/default.asp}.

\bibitem{web:fightcode} \emph{Fightcode}. Pridobljeno 6.6.2016 iz,
  \url{http://fightcodegame.com/}.

\bibitem{web:codeschool} \emph{Codeschool}. Pridobljeno 6.6.2016 iz,
  \url{https://www.codeschool.com/}.

\bibitem{wiki:k12} Wikipedia contributors, \emph{Education in the
    United States}, Wikipedia, The Free Encyclopedia. Pridobljeno
  5.6.2016 iz,
  \url{https://en.wikipedia.org/wiki/Education_in_the_United_States#K.E2.80.9312_education}.

\bibitem{web:moodle_site} \emph{Moodle}. Pridobljeno 6.6.2016 iz,
  \url{https://moodle.org/}.

\bibitem{web:code.org:promote} \emph{Code.org, Promote}, Pridobljeno
  22.6.2016 iz, \url{https://code.org/promote}

\bibitem{web:code.org:about} \emph{Code.org, About us}, Pridobljeno
  22.6.2016 iz, \url{https://code.org/about}

\bibitem{web:code.org} \emph{Code.org}, Pridobljeno
  22.6.2016 iz, \url{https://code.org}

\bibitem{web:code.org:studio} \emph{Code studio}, Pridobljeno
  27.6.2016 iz, \url{https://studio.code.org}

\bibitem{web:codeacademy} \emph{Codeacademy}, Pridobljeno 6.9.2016 iz,
  \url{https://www.codecademy.com}

\bibitem{web:edublogger} Elliott Bristow, \emph{Gaming in education: Gamification}. Pridobljeno 6.6.2016 iz,
  \url{https://www.theedublogger.com/2015/01/20/gaming-in-education-gamification/}.

\bibitem{web:scratch} \emph{Scratch}, Pridobljeno 15.9.2016 iz,
  \url{https://scratch.mit.edu/}

\bibitem{web:scratch:about} \emph{Scratch, About}, Pridobljeno 15.9.2016 iz,
  \url{https://scratch.mit.edu/about/}

\bibitem{web:replIT} \emph{Repl.it}, Pridobljeno 18.9.2016 iz,
  \url{https://repl.it/}

\bibitem{web:freecodecamp} \emph{Freecodecamp}, Pridobljeno 18.9.2016 iz,
  \url{https://www.freecodecamp.com}

\bibitem{web:tutorialspoint} \emph{Tutorialspoint}, Pridobljeno 18.9.2016 iz,
  \url{http://www.tutorialspoint.com/}

\bibitem{web:tutorialspoint:codingground} \emph{Tutorialspoint,
    Codingground}, Pridobljeno 18.9.2016 iz,
  \url{http://www.tutorialspoint.com/codingground.htm}

\bibitem{web:codecombat} \emph{Code combat}, Pridobljeno 20.6.2016 iz,
  \url{https://codecombat.com}

\bibitem{web:codecombat:about} \emph{Code combat, About}, Pridobljeno
  20.6.2016 iz, \url{https://codecombat.com/about}

\bibitem{web:thimble} \emph{Thimble by mozilla}, Pridobljeno
  30.6.2016 iz, \url{https://thimble.mozilla.org}


\bibitem{web:codingame} \emph{Codingame}, Pridobljeno 28.6.2016 iz,
  \url{https://www.codingame.com/home}









%%codingground pride pozneje
\bibitem{web:codingground} Codingground, \emph{Execute python
    online}. Pridobljeno 6.6.2016 iz,
  \url{http://www.tutorialspoint.com/execute_python_online.php}.





% \bibitem{bilten01}
%   \emph{Tematska številka Biltena E-šolstvo}, \\
%   Bilten E-šolstva, Številka: 2010/1\\
%  elektronski vir: \url{http://projekt.sio.si/wp-content/uploads/sites/8/2015/01/E-solstvo_BILTEN_2010-1_screen.pd}\\
% (E-središče v okviru projekta E-šolstvo, 2010)

% \bibitem{test}
%  CPI, \emph{O CPI}, \\
%   \url{http://www.cpi.si/o-cpi.aspx},\\
%   Obiskano: \emph{\datum}

%  \bibitem{esolska_torba}
%   SIO, \emph{E-šolska torba}, \\
%    \url{http://projekt.sio.si/e-solska-torba/},\\
%    Obiskano: \emph{\datum}

%  \bibitem{sio_arhiv}
%   SIO, \emph{Repozitorij gradiv}, \\
%    \url{http://portal.sio.si/no_cache/sio/gradiva/repozitorij_gradiv_trubar/},\\
%    Obiskano: \emph{\datum}

%  \bibitem{stran_execute}
%   \emph{Spletna stran execute}, \\
%    \url{http://execute.fnm.uni-mb.si},\\
%    Obiskano: \emph{\datum}


% \bibitem{iUčbeniki}
%   Andreja Čuk ... [et al.], \\
%   \emph{SLOVENSKI i-učbeniki} [elektronski vir], \\
%   način dostopa: \url{http://www.zrss.si/pdf/slovenski-i-ucbeniki.pdf}\\
%   (Zavod republike slovenije za šolstvo, 2014)


% \bibitem{dipl_simulacije1}
%   Vera Kožuh \\
%   \emph{Diplomska seminarska naloga - Simulacije z računalnikom pri
%     pouku fizike v osnovni šoli}, \\
%  elektronski vir: \url{http://splet-stari.fnm.uni-mb.si/pefmb_old/didgradiva/diplome/kozuh/}\\
% (Pedagoška fakulteta - Oddelek za fiziko, Maribor 1999)


% \bibitem{informatika_2010}
%   Avtor, \\
%   \emph{Naslov} \\
% (Založba, Mesto 2006)



\end{thebibliography}


  %%% Local Variables:
  %%% mode: latex
  %%% TeX-master: "diploma"
  %%% End:
